\section{Results}

%An outline for results  (under consideration)
%
%\begin{itemize}
%
%\item Form two separate subsections, one for the GTEx Version 4 data and the other for the single cell Zeisel data.
%
%\item For GTEx data, give a figure comprising of 4 Structure plots for different $K$s, may be $2,5,10,15$. Fix the thinning parameter $p_{thin}$ to say $0.0001$.  Also record the log likelihoods (Bayes factors) for each of the 4 models, as reported by \textbf{maptpx}. 
%
%\item Have one figure showing the robustness of the clustering method on the thinning parameter $p_{thin}$. Fix $k=10$ and vary $p_{thin}$ to be $0.0001$, $0.001$ and $0.01$. 
%
%\item One t-SNE plot for GTEx samples (with and without admixture in the same plot). Should this be in results or in discussions? Also the t-SNE probably would require an electronic supplemental file as I would need the \textbf{qtlcharts} highlighting for those plots. 
%
%\item The GTEX brain samples Structure plot  for $K=4$ that shows the neuron cell types in brain cerebellum and cerebellar hemisphere. That is to show that the clusters are driven by cell types.
%
%\item Gene annotations for the GTEx significant genes (for brain) and also for the general set up (to decide on which $K$ to fix). Use Bayes Factor?
%
%\item The Structure plot for Zeisel single cell data. again Multiple $K=2,5,7,10$. 
%
%\item Gene annotations for the Zeisel single cell data. Need to choose the optimal $K$. Use Bayes Factor?
%
%\item t-SNE plot of the admixture proportions??..Is that required? Depends on how we present t-SNE. If this goes to discussion, we will avoid it here
%
%\end{itemize}

We begin by illustrating our method on the tissue level data from the  GTEx project (V6 dbGaP accession phs000424.v6.p1, release date: Oct 19, 2015, \url{http://www.gtexportal.org/home/}) read counts data.  RNA-seq data was obtained from $8555$ samples collected from $450$ donors across $51$ tissues and $2$ cell-lines. We collected a set of $16069$ cis-genes that satisified quality check (gene list available in \url{https://github.com/stephenslab/count-clustering/blob/master/utilities/gene_names_GTEX_V6.txt}).  \\[1 pt]

 \textbf{Fig \ref{fig:fig1}(a)} presents the Structure plot for admixture model fit for $K=15$. The Structure plot highlights the similarity among the samples coming from the same tissue and also tells us which tissues have similar patterns of gene expression. It is seen that the different Brain tissues seem to cluster together, the same being true for the arteries (Artery-aorta, Artery-tibial and Artery-coronary). Interestingly, Nerve Tibial  and Adipose tissues (Adipose Subcutaneous and Adipose Visceral (Omentum)) also seem to have similar clustering patterns. \\[1 pt]
 
 As observed from the Structure plot, some tissues seem to be assigned to separate clusters for $K=15$, (e.g: Pancreas, Whole Blood), but other tissues are represented as an "admixed" version of multiple clusters (e.g: Thyroid). But in these latter cases, the samples coming from the same tissue all seem to be cluster together as they have very similar patterns of "admixing" of different clusters. A different way of visualizing the results, which highlights the clustering and separation of the different tissues on a 2D projection space (see \textbf{Supplementary Fig 1 [url]}). \\[1 pt]
 
To biologically interpret the clusters in  \textbf{Fig \ref{fig:fig1}}, we performed cluster annotation (see Methods and Materials). \textbf{Tab ~\ref{tab:tab1}} presents the gene IDs, names and a short summary of the functions of top $3$ driving genes for each cluster. The cluster annotation in \textbf{Tab~\ref{tab:tab1}} highlights tissue specific functions and pathways. We observe that \textit{PRSS1}(protease serine 1), \textit{CPA1} (carboxypeptidase) and \textit{PNLIP} (pancreatic lipase) are the  top three genes that drive the cluster separating Pancreas from the other tissues. Similarly, \textit{HBB} (hemoglobin, beta), \textit{HBA2} (hemoglobin, alpha 2) and \textit{HBA1} (hemoglobin, alpha 1) seem to be the top three genes that distinguish Whole Blood and drive a separate cluster from the rest. \\[2 pt]

%A field of very active interest in recent times is to estimate the proportion of different cell types in different tissues. Marker based approaches are usually adopted to validate for different cell types and get a sense of the abundance of different cell types in the tissue samples \cite{Grun2015} \cite{Palmer2005}. The admixture model is a marker free method to obtain clusters driven by cell types. 

Although global analysis of all tissues is useful for highlighting major structure in the data, it is less well suited to identifying structure within tissues or among similar tissues. Therefore, we ran the analysis on samples from a particular tissue or similar tissues. \textbf{Fig \ref{fig:fig1}(b)} shows the Structure plot for $K=4$ on just the Brain samples. Clearly, we see additional patterns across the Brain tissues in this plot compared to the global Structure plot in \textbf{Fig \ref{fig:fig1}(a)}. Brain Cerebellum and Cerebellar hemisphere seem to have   $80-85 \%$ membership  proportion in green cluster which seems distinctive . Recent stereological approaches have shown that rat cerebellum contains $> 80 \%$ neurons (Herculano-Houzel and Lent 2005) \cite{Houzel2005}, much higher than other parts of the brain. We performed cluster annotation and observed that the pivotal genes that separated out the green cluster in brain cerebellum and cerebellar hemisphere are SNAP25 (synaptosomal-associated protein, 25kDa), ENO2 (enolase 2- gamma, neuronal) and CHGB (chromogranin B),  all of which are associated with neuronal activities (see \textbf{Supplementary Table 1}). Therefore, the green cluster does seem to be driven by neuronal cell types, though not completely determined by them (since this cluster is almost non-existent in other parts of the Brain).\\[3 pt]

We aimed to assess the cluster performance of the graded membership method on lower coverage data as is observed in single cell RNA seq (scRNA seq) data. To compare the cluster performance between bulk-RNA coverage and single cell-RNA coverage, we thinned the GTEx reads data. If $c_{ng}$ is the counts of number of reads mapping to gene $g$ for sample $n$ for the original data, then the thinned counts are given by 

$$ t_{ng} \sim Bin (c_{ng}, p_{thin})$$

where $p_{thin}$ represents the proportional coverage of a single cell RNA-seq data compared to the bulk RNA-seq data. We compared the total library size between the GTEx tissue level data and the single cell data due to Jaitin \textit{et al} \cite{Jaitin2014} and observed that $p_{thin} =0.0001$. We then used this thinning parameter and fitted the clustering model on $t_{ng}$s. \textbf{Fig \ref{fig:fig1}(c)} presents the Structure plot for $K=15$ for the thinned data. Many of the features from \textbf{Fig \ref{fig:fig1}(a)} are preserved even after thinning, for instance the Brain tissues clustering together, Adipose tissues and Nerve Tibial also showing similar patterns. This implies that the clustering model is robust to the coverage of the data. \\[1 pt]

We next sought to demonstrate more quantitatively the utility of the model based clustering compared to other non model based clustering methods such as hierarchical clustering. In \textbf{Fig \ref{fig:fig2}}, we consider every pair of tissues from the list of tissues in GTEX with number of samples $> 50$. Then we generated a set of $50$ samples randomly drawn from the pooled set of samples coming from these two tissues and then observed whether the hierarchical and the admixture were separating out samples coming from the two different tissues. The same remains true for  thinned GTEx data under different choices of thinning parameters. Check  \textbf{Fig \ref{fig:figS2}} for demonstration.  \\[1 pt]

We applied the model on a couple of single cell datasets due to Jaitin \textit{et al} \cite{Jaitin2014} and Deng \textit{et al} \cite{Deng2014}.  Jaitin \textit{et al} sequenced around $4000$ single cells from mouse spleen, where the cells were a heterogeneous mix enriched for expression of CD11c marker.  The aim of their study was to separate out the B cells, NK cells, pDCs and monocytes.  However the biological effect in their study was completely confounded with the amplification and sequencing batches.  \textbf{Fig \ref{fig:fig3}} (\textit{top panel}) presents the Structure plot  for $K=7$ for the Jaitin \textit{et al} data with the samples arranged by their amplification batch (which was a refinement of the sequencing batch). This highlights the need for caution regarding interpreting Admixture results or any clustering results, as there is a possibility of technical effects driving the clusters instead of true biological effects. There has been a growing concern among biostatisticians today about how to deal with batch effects \cite{Leek2010} \cite{Hicks2015}. \\[1 pt]

Deng \textit{et al} analyzed the allelic expression of individual cells from oocyte to blastocyst stages of CAST/EiJ � C57BL/6J mouse preimplantation development.  \textbf{Fig \ref{fig:fig3}} (\textit{bottom panel}) presents the Structure plot for  $K=4$ on the single cells from different development stages starting from zygote, to early/mid/late 2 cells, 8 cells, 16 cells, early/mid blastocyst to finally late blastocyst. and shows the continuity of the clustering patterns with respect to progress of embryo development. Some developmental phases are represented by a single cluster, for example- \textit{zygote/early2cell}, \textit{mid/late2cell}, \textit{8cell/16cell} and \textit{lateblast}, while some developmental phases can be written as a mix of two clusters representing the adjacent phases. For example, \textit{4cell} phase can be written as a mix of the cluster representing the \textit{mid/late2cell} and the \textit{8cell/16cell} phase.\\[1 pt]


%, 
%We find that admixture model is more successful in separating out different tissues in general, compared to the hierarchical clustering technique. The admixture model is essentially a count based modeling approach and seems to handle low counts and zero counts much better than the hierarchical method which is a more general approach to clustering. Since the RNA-seq data and in particular scRNA-seq data have lots of low counts and zero counts, the admixture model seems to be more suited for such data compared to hierarchical clustering method. 

%Currently there is a lot of interest in single cell sequencing as it is more informative about individual cell expression profiles compared to the RNA-seq on tissue samples. We were curious to see how stable the Admixture results are if the GTEx RNA-seq data is viewed at the scale of a single cell data. We achieve the latter by thinning the GTEx data under thinning parameter $p_{thin}=0.0001$ which is the order of scale obtained by dividing the total library size of the Jaitin \textit{et al} \cite{Jaitin2014} with respect to the library size of the GTEx V4 read counts data. We fitted the admixture model for $K=12$ on the thinned data and the Structure plot for the fitted model is presented in \textbf{Fig \ref{fig:fig4}}. It seems that most of the features observed in \textbf{Fig \ref{fig:fig1}} seem to be retained, for instance- the Brain samples clustering together, Whole blood and Testis forming separate clusters, Muscle skeletal and Heart tissue samples showing very similar patterns etc. However, thinning indeed shrinks the small differences across tissues and makes it more difficult to distinguish between tissues, as evident from the comparative study of hierarchical and admixture models, analogous to \textbf{Fig \ref{fig:fig3}}, for thinned data with thinning parameters $p_{thin}=0.001$ and $p_{thin}=0.0001$ in  \textbf{Fig \ref{fig:figS2}}. One can see that with thinning, the performance of admixture model in separating the tissues deteriorates but encouragingly, it seems that admixture does outperform the hierarchical clustering even under thinned data. \\[3pt]

 








