\section{Methods and Materials}

\subsection{Data preprocessing}

RNA-seq experiments provide us with a set of FASTQ files that contain the nucleotide sequence of each read and a quality score at each position, which can be mapped to  reference genome or exome or transcriptome. The output of this mapping is usually saved in a SAM/BAM file using SAMtools  \cite{Li2009}, a task primarily accomplished by \textit {htseq-counts}  by Sanders et al  2014 \cite{Sanders2014} or \textit{featureCounts}  [ R package \textbf{Rsubread} ] by Liao et al 2013 \cite{Liao2013}.  RNA-seq raw counts are the basis of all statistical workflows, be it exploration or differential expression analysis [\textbf{edgeR} \cite{Robinson2010}, \textbf{limma} \cite{Ritchie2015} ]. There is a growing trend to make the analysis ready raw counts tables openly accessible for statistical analysis. ReCount is a online site that hosts RNA-seq gene counts datasets from 18 different studies \cite{Frazee2011} along with relevant metadata. Such gene counts datasets are the inputs for our clustering algorithm. \\[2 pt]

 In the preprocessing step before applying our method, we remove the genes with 0 or same count of matched reads across all samples (non-informative genes), any sample or gene  with NA values of reads and  ERCC spike-in controls,  as the latter may create bias due to their typical very high expression (number of reads mapped to them).  For illustration, we applied our method GTEx Version 6 tissue level gene counts data \cite{GTEX2013} and on a couple of single cell data due to Zeisel \textit{et al} \cite{Zeisel2015} and Jaitin \textit{et al} \cite{Jaitin2014}. 
 
% If $C_{ng}$ is the gene count for $g$ th gene in tissue sample $n$, then we define the thinned counts as 
%
%$$ c_{ng}  \sim Bin(C_{ng}, p_{thin} )  $$
%
%where $p_{thin}$ is the thinning probability. W chose $p_{thin}$ to be of the order of the ratio of the total number of reads mapped to a single cell experiment (in this case Zeisel et al (2015) data for instance) and the total number of reads in the GTEx dataset, which turned out to be approximately 0.0001. To check for robustness of our clustering algorithm, we varied $p_{thin}$ to be $0.01, 0.001, 0.0001$ (see Fig ).  
%

\subsection{Model overview}

We use a topic model approach due to Matt Taddy (package \textbf{maptpx}) to perform the clustering of the samples based on RNA-seq reads data \cite{Taddy2012}. Let us denote the gene counts matrix as $C_{N \times G}$ where $N$ is the total number of samples (tissue/single cell) and $G$ is the number of genes.  We assume that the row vector of counts for each sample $n$ across the genes follows a multinomial distribution.

$$ c_{n*} \sim Mult(c_{n..}, p_{n*}) $$

where $c_{n*}$ is the count vector for the $n$ th sample, $c_{n..}$ is the sum of the counts in the vector $c_{n*}$, and $p_{n*}$ is the probability that a read coming from sample $n$ would get assigned to one of the $G$ genes. The idea here is that this read could be coming from some cell type for the tissue level expression study (or from some cell cycle phase for the single cell case study) and its probability of getting assigned to some gene $g$ will depend on which cell type (cell cycle phase) it comes from. In general, we may assume that the read is coming from one of the several (say $K$) underlying classes/groups, which are not observed. Denote  the probability that the sample is coming from the $k$ th subgroup by $q_{nk}$ ($q_{nk} \geq 0$ and $\sum_{k=1}^{K} q_{nk} =1$ for each $n$) and then given that the sample is coming from the $k$th subgroup, the probability of a read being matched to the $g$th gene is given by $\theta_{kg}$ ($\theta_{kg} \geq 0$ and $\sum_{g=1}^{G} \theta_{kg} =1$ for $k$th subgroup). Then one can write 

$$ p_{ng} = \sum_{k=1}^{K} q_{nk}\theta_{kg}   \hspace{1 cm}  \sum_{k=1}^{K} q_{nk}=1 \hspace{1 cm} \sum_{g=1}^{G} \theta_{kg}=1 $$

This model has in all $N \times (K-1) + K \times (G-1)$ many unconstrained parameters, which is much smaller than the $N \times G$  data values of counts. Usually $K << min \{N,G \} $, $N$ in the region of $100$s to $1000$s  and $G$ ranging from $10,000$ to $50,000$ (depending on the species the RNA-seq data is coming from and whether the set of genes recorded include non-protein genes or not). To estimate the model, a Maximum a posteriori (MAP) based approach is used (see Taddy 2012 \cite{Taddy2012}).

% It assumes the priors
%
%$$ q_{n*} \sim Dir ( \frac{1}{K}, \frac{1}{K}, \cdots, \frac{1}{K} ) $$
%$$ \theta_{k*} \sim Dir(\frac{1}{KG}, \frac{1}{KG}, \cdots, \frac{1}{KG} ) $$
%
%For better estimation stability, the usual parameters of the model are converted to natural exponential family parameters to which one can apply the EM algorithm ). The value of the Bayes factor for the model with $K$ clusters compared to the model with 1 cluster, is recorded for each $K$, and the optimal $K$ is chosen by running the clustering method for different choices of $K$ and then choosing the one with maximum Bayes factor. The two main outputs from this method are the $Q_{N \times K}$ topic proportion matrix  and $F_{K \times G}$ relative gene expression for each cluster.

\subsection{Visualization}

For each $n$, $q_{nk}$'s which will give an idea about the relative abundance of individual subgroups (which may be driven by cell functional groups or cell types) represented in the sample (single cell or tissue respectively). If two samples $n$ and $n^{'}$ are very close, say both coming from the same tissue for the tissue level data, then we expect $q_{n*}$ and $q_{n^{'}*}$ to be very close too. A nice way to visualize the amount of relatedness among the samples is through the Structure plot due to Pritchard Lab, which is a popular tool to visualize the admixture patterns in population genetics based on SNP/ microsatellite data \cite{Pritchard2000} \cite{Raj2014}. The Structure plot  assigns a color to each of the subgroups and then presents a vertical barplot for each individual, which is fragmented by the subgroup proportions and colored accordingly. If the colored patterns of two bars are similar, then the two samples must be closely related. \\[1 pt]
Another visualizing tool we recommend is t-distributed Stochastic Neighbor Embedding (t-SNE) due to Laurens van der Maaten, which is well-suited for visualizing the high dimensional datasets on 2D, preserving the relative distance between samples in high dimension to a fair extent in 2D \cite{Maaten2008} \cite{Maaten2014}. t-SNE provides some sense about which samples are closer to each other when the data is projected on 2D. But on the flipside, it is not a clustering tool and unlike Structure plot, does not show the relative abundance patterns of different subgroups in the sample. However, both Structure plot and t-SNE give a lot more interpretable visualization of the clustering compared to the heatmap and hierarchical clustering (see Results for illustration).

\subsection{Cluster annotation}

A question of considerable biological interest is which genes are significantly differentially expressed across the clusters, or in other words, which genes are driving the clustering. To answer this, we fix each gene and then look at the KL divergence matrix of one cluster/subgroup $k$ relative to other cluster/subgroup $k^{'}$, which we call $KL^{g}_{K \times K}$. This matrix is symmetric and has all diagonal elements $0$ as the divergence of a cluster with respect to itself is $0$. Next we define the divergence measure for gene $g$ as 

$$ Div(g) = \underset{k}{max} \; \underset{l \neq k}{min} \; KL^{g} [k, l] $$

$$ K_{div}(g) = arg \;  \underset{k}{max} \; \underset{l \neq k}{min}  \; KL^{g} [k, l] $$


The higher the divergence measure, the more significant is the role of the gene in the clustering. We choose a small subset of around 50-100 genes with highest values of $Div(g)$ and put the gene in the $K_{div}(g)$ th cluster/subgroup. Then we perform gene annotations for the top genes in each subgroup using \textbf{mygene} R Bioconductor package \cite{Thompson2014}. We observe if the significant genes in a particular subgroup/cluster are associated with some specific biological functionality. This would indicate if the subgroups are actually biologically relevant or not. For instance, for GTEx tissue sample data, if the clusters are indeed driven by cell types, then the top genes for these clusters will probably be associated with proteins related to  functions for that particular cell type.


\medskip
















 









