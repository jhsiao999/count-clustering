\section{Methods and Materials}

% We also remove spike-in control genes,  as the latter may create bias due to their typically having high number of reads mapped to them \cite{Jiang2011}. 

%RNA-seq experiments provide us with a set of FASTQ files that contain the nucleotide sequence of each read and a quality score at each position, which can be mapped to  reference genome or exome or transcriptome. The output of this mapping is usually saved in a SAM/BAM file using SAMtools  \cite{Li2009}, a task primarily accomplished by \textit {htseq-counts}  by Sanders et al  2014 \cite{Sanders2014} or \textit{featureCounts}  [ R package \textbf{Rsubread} ] by Liao et al 2013 \cite{Liao2013}.  RNA-seq raw counts are the basis of all statistical workflows, be it exploration or differential expression analysis [\textbf{edgeR} \cite{Robinson2010}, \textbf{limma} \cite{Ritchie2015} ]. There is a growing trend to make the analysis ready raw counts tables openly accessible for statistical analysis. ReCount is a online site that hosts RNA-seq gene counts datasets from 18 different studies \cite{Frazee2011} along with relevant metadata. Such gene counts datasets are the inputs for our clustering algorithm. \\[2 pt]
%

% In the preprocessing step before applying our method, we remove the genes with 0 or same count of matched reads across all samples (non-informative genes), any sample or gene  with NA values of reads and  ERCC spike-in controls,  as the latter may create bias due to their typical very high expression (number of reads mapped to them).  For illustration, we applied our method GTEx Version 6 tissue level gene counts data \cite{GTEX2013} and on a couple of single cell data due to Zeisel \textit{et al} \cite{Zeisel2015} and Jaitin \textit{et al} \cite{Jaitin2014}. 
 
% If $C_{ng}$ is the gene count for $g$ th gene in tissue sample $n$, then we define the thinned counts as 
%
%$$ c_{ng}  \sim Bin(C_{ng}, p_{thin} )  $$
%
%where $p_{thin}$ is the thinning probability. W chose $p_{thin}$ to be of the order of the ratio of the total number of reads mapped to a single cell experiment (in this case Zeisel et al (2015) data for instance) and the total number of reads in the GTEx dataset, which turned out to be approximately 0.0001. To check for robustness of our clustering algorithm, we varied $p_{thin}$ to be $0.01, 0.001, 0.0001$ (see Fig ).  
%

\subsection{Model overview}

We assume the RNA-seq data have been summarized by a table of counts $C_{N \times G} = (c_{ng})$, where $c_{ng}$ is the number of reads from sample $n$ mapped to gene (or transcript) $g$ \cite{Oshlack2010}.  We remove genes $g$ with all zero counts ($c_{ng}=0$ for all $n$), and
use the {\tt maptpx} R package \cite{Taddy2012} to fit the grade of membership (GoM) model, also known as ``Latent Dirichlet Allocation" (LDA). 
This model assumes the RNA-seq counts for each sample follow a multinomial distribution
\begin{equation}
c_{n\cdot} \sim \text{Mult}(c_{n+}, p_{n\cdot})
\end{equation}
where $c_{n\cdot}$ denotes the count vector for the $n$th sample, $c_{n+} := \sum_g c_{ng}$, and $p_{n\cdot}$ is a probability vector (non-negative entries summing to 1) whose $g$th element represents the relative expression of gene $g$ in sample $n$. 
The model further assumes that 
\begin{equation}
p_{ng} = \sum_{k=1}^{K} q_{nk}\theta_{kg}    
\end{equation}
where $q_{n\cdot}$ is a probability vector whose $k$th element represents the grade of membership of
sample $n$ in cluster $k$, and $\theta_{k\cdot}$ is a probability vector whose $g$th element represents
the relative expression of gene $g$ in cluster $k$. The {\tt maptpx} package fits this model using an EM algorithm to perform Maximum a posteriori (MAP)  estimation of the parameters $q$ and $\theta$. See \cite{Taddy2012} for details.

%This model has $N \times (K-1) + K \times (G-1)$ parameters, which is much smaller than the $N \times G$  data values of counts. Usually for RNA-seq samples $N$ varies in the region of $100$s to $1000$s  and $G$ ranges from $10,000$ to $50,000$ (depending on the underlying species and the types of genes tokenized) and $K << \{N,G \}$. 

% It assumes the priors
%
%$$ q_{n*} \sim Dir ( \frac{1}{K}, \frac{1}{K}, \cdots, \frac{1}{K} ) $$
%$$ \theta_{k*} \sim Dir(\frac{1}{KG}, \frac{1}{KG}, \cdots, \frac{1}{KG} ) $$
%
%For better estimation stability, the usual parameters of the model are converted to natural exponential family parameters to which one can apply the EM algorithm ). The value of the Bayes factor for the model with $K$ clusters compared to the model with 1 cluster, is recorded for each $K$, and the optimal $K$ is chosen by running the clustering method for different choices of $K$ and then choosing the one with maximum Bayes factor. The two main outputs from this method are the $Q_{N \times K}$ topic proportion matrix  and $F_{K \times G}$ relative gene expression for each cluster.

\subsection{Visualizing Results}

We visualize results using a ``Structure plot" \cite{Rosenberg2002}, 
which is named for its widespread use in visualizing the
results of the ``structure" software \cite{Pritchard2000} in population genetics.
The Structure plot represents each GoM vector $q_{n\cdot}$
as  a vertical stacked barchart, with bars of different colors representing membership proportion in each cluster (e.g.~Figure \ref{fig:fig1}). If the colored patterns of two bars are similar, then the two samples have similar membership proportions.  The Structure plot is particularly helpful when external information is available on each sample that can be used to order or group the samples in an informative way.

We have also found it useful to visualize results using t-distributed Stochastic Neighbor Embedding (t-SNE), which is a method for visualizing high dimensional datasets by placing them in a two dimensional space, attempting to preserve the relative distance between nearby samples \cite{Maaten2014,Maaten2008}. t-SNE tends to place samples with similar membership proportions together in the two-dimensional plot, forming visual ``clusters" that can be identified by eye (e.g. Supplementary Figure 1). This may be particularly helpful in settings where no external information is available to aid in making an informative Structure plot. 


\subsection{Cluster annotation}

To help biologically interpret the clusters, we developed a method to identify which genes are most distinctively differentially expressed in each cluster.  Specifically, for each cluster $k$ we measure the distinctiveness of gene $g$ with respect to any other cluster $l$ using
\begin{equation}
\KL^{g} [k,l] : = \theta_{kg} \; log \frac{\theta_{kg}}{\theta_{lg}} + \theta_{lg} - \theta_{kg},
\end{equation}
which is the Kullback--Leibler divergence of the Poisson distribution with parameter $\theta_{kg}$ to the Poisson distribution with parameter $\theta_{lg}$. 
For each cluster $k$, we then define the distinctiveness of gene $g$ as 
\begin{equation}
D^{g}[k] = \underset{l \neq k}{\min} \; \KL^{g} [k, l].
\end{equation}
The higher $D^g[k]$, the larger the role of gene $g$ in distinguishing cluster $k$ from all other clusters. 
Thus, for each cluster $k$ we identify the genes with highest $D^{g}[k]$ as the genes driving the cluster $k$. 
We annotate the genes driving each cluster with biological functions using the {\tt mygene} R Bioconductor package  \cite{Thompson2014}.

\subsection{Comparison with hierarchical clustering}

Distance based hierarchical clustering methods are the most commonly used clustering techniques for gene expression data. To compare between the grade of membership model and the distance based hierarchical clustering algorithm, we used both methods to samples from pairs of tissues from the GTEX project, and assessed 
which methods separated samples according to tissue.  For each pair of tissues  we randomly selected $50$ samples from the pool of all samples coming from these tissues. 
For the hierarchical clustering approach we cut the dendogram at $K=2$, and checked whether or not this cut partitions the samples into the two tissue groups. 
(We applied hierarchical clustering using Euclidean distance, with both complete and average linkage; results were similar and so we showed results only for complete linkage.) 
For the model-based approach we analysed the data with $K=2$, and sort the samples by their membership in cluster 1. We then partitioned the samples at the point of the steepest fall in this membership, and again we check whether this cut partitions the samples into the two tissue groups.

Figure \ref{fig:fig2} shows, for each pair of tissues, whether each method successfully partitioned the samples into the two tissue groups using these approaches.
%For instance, for GTEx tissue sample data, if the clusters are indeed driven by cell types, then the top genes for these clusters will probably be associated with proteins related to  functions for that particular cell type.
%

%we fix each gene and then look at the KL divergence matrix of one cluster/subgroup $k$ relative to other cluster/subgroup $k^{'}$, which we call $KL^{g}_{K \times K}$. This matrix is symmetric and has all diagonal elements $0$ as the divergence of a cluster with respect to itself is $0$. 




%Then we perform gene annotations for the top genes in each subgroup 


















 









