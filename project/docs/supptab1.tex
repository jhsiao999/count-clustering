\section{Supplementary Table 1}
\begin{table}[htp]
\caption{Cluster Annotations GTEx V6 data (with top gene summaries). \label{tab:supptab1}} 
\begin{center}
\begin{tabular}{|p{0.7in}|p{0.5in}|p{1.4in}|p{3.6in}|} 
\hline
Cluster & Top Driving \qquad Genes & Gene names  & Gene Summary \\
\hline

 \multirow{3}{4em}{\footnotesize{cluster 1, purple} } &  \small{\textit{FABP4}} & \footnotesize{ fatty acid binding protein 4, adipocyte} & \scriptsize{FABP4 encodes the fatty acid binding protein found in adipocytes, roles include fatty acid uptake, transport, and metabolism}\\ 
 					      & \small{\textit{APOD}} & \footnotesize{apolipoprotein D} & \scriptsize{encodes a component of high density lipoprotein that has no marked similarity to other apolipoprotein sequences, closely associated with lipoprotein metabolism.}\\
					      & \small{\textit{PLIN1}} & \footnotesize{perilipin} 1 & \scriptsize{coats lipid storage droplets in adipocytes, thereby protecting them until they can be broken down by hormone-sensitive lipase.} \\
 \hline
 \multirow{3}{4em}{\footnotesize{cluster 2, light purple} } & \small{\textit{MYH11}} & \footnotesize{myosin, heavy chain 11, smooth muscle} & \scriptsize{functions as a major contractile protein, converting chemical energy into mechanical energy through the hydrolysis of ATP.} \\
 					    & \small{\textit{ACTA2}} & \footnotesize{actin, alpha 2, smooth muscle, aorta} & \scriptsize{protein encoded by this gene belongs to the actin family of proteins, which are highly conserved proteins that play a role in cell motility, structure and integrity, defects in this gene cause aortic aneurysm familial thoracic type 6.}\\
					    &  \small{\textit{ACTG2}} & \footnotesize{actin, gamma 2, smooth muscle, enteric} & \scriptsize{encodes actin gamma 2; a smooth muscle actin found in enteric tissues, involved in various types of cell motility and in the maintenance of the cytoskeleton. }\\
 \hline
 \multirow{3}{4em}{\footnotesize{cluster 3, red}} & \small{\textit{MBP}} & \footnotesize{myelin basic protein} & \scriptsize{major constituent of the myelin sheath of oligodendrocytes and Schwann cells in the nervous system} \\
 					    & \small{\textit{GFAP}} & \footnotesize{glial fibrillary acidic protein} & \scriptsize{encodes one of the major intermediate filament proteins of mature astrocytes, mutations casuses Alexander disease.} \\
					    & \small{\textit{SNAP25}}  & \footnotesize{synaptosomal-associated protein, 25kDa} & \scriptsize{this gene product is a presynaptic plasma membrane protein involved in the regulation of neurotransmitter release.}\\
\hline
 \multirow{3}{4em}{\footnotesize{cluster 4, light red}} & \small{\textit{PRM2}} & \footnotesize{protamine 2} & \scriptsize{Protamines are the major DNA-binding proteins in the nucleus of sperm} \\
 					      & \small{\textit{PRM1}} & \footnotesize{protamine 1} & \scriptsize{Protamines are the major DNA-binding proteins in the nucleus of sperm}  \\
					      & \small{\textit{PHF7}} & \footnotesize{PHD finger protein 7} & \scriptsize{This gene is expressed in the testis in Sertoli cells but not germ cells, regulates spermatogenesis.} \\
\hline					      					      
 \multirow{3}{4em}{\footnotesize{cluster 5, blue}} & \small{\textit{TG}} & \footnotesize{thyroglobulin} & \scriptsize{thyroglobulin produced predominantly in thyroid gland, synthesizes thyroxine and triiodothyronine, associated with Graves disease and Hashimotot thyroiditis.} \\
 					     & \small{\textit{LIPF}} & \footnotesize{lipase, gastric} & \scriptsize{encodes gastric lipase, an enzyme involved in the digestion of dietary triglycerides in the gastrointestinal tract, and responsible for 30 $\%$ of fat digestion processes occurring in human.} \\
					     & \small{\textit{PGC}} & \footnotesize{progastricsin (pepsinogen C)} & \scriptsize{encodes an aspartic proteinase that belongs to the peptidase family A1. The encoded protein is a digestive enzyme that is produced in the stomach and constitutes a major component of the gastric mucosa,  associated with susceptibility to gastric cancers.}\\					     
 \hline
 \multirow{3}{4em}{\footnotesize{cluster 6, sky blue}} & \small{\textit{KRT10}} & \footnotesize{keratin 10, type I} & \scriptsize{encodes a member of the type I (acidic) cytokeratin family, mutations associated with epidermolytic hyperkeratosis.}\\
 					    &  \small{\textit{KRT1}} & \footnotesize{keratin 1, type II} & \scriptsize{specifically expressed in the spinous and granular layers of the epidermis with family member KRT10 and mutations in these genes have been associated with bullous congenital ichthyosiform erythroderma.} \\
					    & \small{\textit{KRT2}} & \footnotesize{keratin 2, type II} & \scriptsize{expressed largely in the upper spinous layer of epidermal keratinocytes and mutations in this gene have been associated with bullous congenital ichthyosiform erythroderma.}\\
 \hline
 \multirow{3}{4em}{\footnotesize{cluster 7, orange}} & \small{\textit{FN1}}  & \footnotesize{fibronectin 1} & \scriptsize{Fibronectin is involved in cell adhesion, embryogenesis, blood coagulation, host defense, and metastasis.} \\
 					      & \small{\textit{COL1A1}} & \footnotesize{collagen, type I, alpha 1} & \scriptsize{Mutations in this gene associated with osteogenesis imperfecta types I-IV, Ehlers-Danlos syndrome type and Classical type, Caffey Disease}. \\
					      & \small{\textit{COL1A2}} & \footnotesize{collagen, type I, alpha 2} & \scriptsize{Mutations in this gene associated with osteogenesis imperfecta types I-IV, Ehlers-Danlos syndrome type and Classical type, Caffey Disease}. \\
\hline
\end{tabular}
\end{center}
\end{table}

\newpage

%
%
%\begin{table}
%\newpage
%\begin{center}
%\begin{tabular}{|c|p{0.5in}|p{1.9in}|p{2.5in}|} 
%\hline
%Cluster & Gene names & Proteins  & Summary \\
%\hline
%
% \multirow{3}{4em}{cluster black (Testis)} & PRM2 & protamine 2 & \small{Protamines are the major DNA-binding proteins in the nucleus of sperm} \\
% 					      & PRM1 & protamine 1 & \small{Protamines are the major DNA-binding proteins in the nucleus of sperm}  \\
%					      & PHF7 & PHD finger protein 7 & \small{This gene is expressed in the testis in Sertoli cells but not germ cells, regulates spermatogenesis.} \\
%\hline					      					      
% \multirow{3}{4em}{cluster light blue (Thyroid, Stomach} & TG & thyroglobulin & \small{thyroglobulin produced predominantly in thyroid gland, synthesizes thyroxine and triiodothyronine, associated with Graves disease and Hashimotot thyroiditis.} \\
% 					     & LIPF & lipase, gastric & \small{encodes gastric lipase, an enzyme involved in the digestion of dietary triglycerides in the gastrointestinal tract, and responsible for 30 $\%$ of fat digestion processes occurring in human.} \\
%					     & PGC & progastricsin (pepsinogen C) & \small{encodes an aspartic proteinase that belongs to the peptidase family A1. The encoded protein is a digestive enzyme that is produced in the stomach and constitutes a major component of the gastric mucosa,  associated with susceptibility to gastric cancers.}\\					     
% \hline
% \multirow{3}{4em}{cluster deep blue (Skin)} & KRT10 & keratin 10, type I & \small{encodes a member of the type I (acidic) cytokeratin family, mutations associated with epidermolytic hyperkeratosis.}\\
% 					    &  KRT1 & keratin 1, type II & \small{specifically expressed in the spinous and granular layers of the epidermis with family member KRT10 and mutations in these genes have been associated with bullous congenital ichthyosiform erythroderma.} \\
%					    & KRT2 & keratin 2, type II & \small{expressed largely in the upper spinous layer of epidermal keratinocytes and mutations in this gene have been associated with bullous congenital ichthyosiform erythroderma.}\\
% \hline
% \multirow{3}{4em}{cluster dark brown (Cells fibroblasts)} & FN1  & fibronectin 1 & \small{Fibronectin is involved in cell adhesion, embryogenesis, blood coagulation, host defense, and metastasis.} \\
% 					      & COL1A1 & collagen, type I, alpha 1 & \small{Mutations in this gene associated with osteogenesis imperfecta types I-IV, Ehlers-Danlos syndrome type and Classical type, Caffey Disease}. \\
%					      & COL1A2 & collagen, type I, alpha 2 & \small{Mutations in this gene associated with osteogenesis imperfecta types I-IV, Ehlers-Danlos syndrome type and Classical type, Caffey Disease}. \\
%\hline	
% \end{tabular}
% \end{center}
%\end{table}
%

\begin{tabular}{|p{0.7in} |p{0.5in}|p{1.4in}|p{3.6in}|}
\hline
Cluster & Top Driving \qquad Genes & Gene namese  & Gene Summary \\
\hline
\multirow{3}{4em}{\footnotesize{cluster 8, light orange}} & \small{\textit{SFTPB}} & \footnotesize{surfactant protein B} & \scriptsize {an amphipathic surfactant protein essential for lung function and homeostasis after birth, muttaions cause pulmonary alveolar proteinosis, fatal respiratory distress in the neonatal period.} \\
 					    & \small{\textit{SFTPA2}} & \footnotesize{surfactant protein A2} & \scriptsize{Mutations in this gene and a highly similar gene located nearby, which affect the highly conserved carbohydrate recognition domain, are associated with idiopathic pulmonary fibrosis.} \\
					    & \small{\textit{SFTPA1}} & \footnotesize{surfactant protein A1} &  \scriptsize{encodes a lung surfactant protein that is a member of C-type lectins called collectins, associated with idiopathic pulmonary fibrosis.} \\
\hline				   
 \multirow{3}{4em}{\footnotesize{cluster 9, green}} & \small{\textit{MYH1}} & \footnotesize{myosin, heavy chain 1, skeletal muscle, adult }& \scriptsize{a major contractile protein which converts chemical energy into mechanical energy through the hydrolysis of ATP.} \\
 					    & \small{\textit{NEB}} & \footnotesize{nebulin} & \scriptsize{encodes nebulin, a giant protein component of the cytoskeletal matrix that coexists with the thick and thin filaments within the sarcomeres of skeletal muscle, associated with recessive nemaline myopathy.} \\
					    & \small{\textit{MYH2}} & \footnotesize{myosin, heavy chain 2, skeletal muscle, adult} & \scriptsize{encodes a member of the class II or conventional myosin heavy chains, and functions in skeletal muscle contraction.} \\
\hline					    
 \multirow{3}{4em}{\footnotesize{cluster 10, l. green}} & \small{\textit{HBB}} & \footnotesize{hemoglobin, beta} & \scriptsize{mutant beta globin causes sickle cell anemia, absence of beta chain/ reduction in beta globin leads to thalassemia.}\\
 					      & \small{\textit{HBA2}} & \footnotesize{hemoglobin, alpha 2} & \scriptsize{deletion of alpha genes may lead to alpha thalassemia.}  \\
					      & \small{\textit{HBA1}} & \footnotesize{hemoglobin, alpha 1} & \scriptsize{deletion of alpha genes may lead to alpha thalassemia.}  \\
\hline				      

 \multirow{3}{4em}{\footnotesize{cluster 11, mint}} &  \small{\textit{NPPA}} & \footnotesize{natriuretic peptide A} & \scriptsize{protein encoded by this gene belongs to the natriuretic peptide family, associated with atrial fibrillation familial type 6.} \\
 					      &  \small{\textit{MYH6}} & \footnotesize{myosin, heavy chain 6, cardiac muscle, alpha} & \scriptsize{encodes the alpha heavy chain subunit of cardiac myosin, mutations in this gene cause familial hypertrophic cardiomyopathy and atrial septal defect 3.} \\
					      &  \small{\textit{ACTC1}} & \footnotesize{actin, alpha, cardiac muscle 1} & \scriptsize{protein encoded by this gene belongs to the actin family, associated with idiopathic dilated cardiomyopathy (IDC) and familial hypertrophic cardiomyopathy (FHC).} \\
\hline			      
 \multirow{3}{4em}{\footnotesize{cluster 12, light yellow}} &  \small{\textit{KRT13}} & \footnotesize{keratin 13, type I} & \scriptsize{protein encoded by this gene is a member of the keratin gene family, associated with the autosomal dominant disorder White Sponge Nevus.}\\
 					      &  \small{\textit{KRT4}} & \footnotesize{keratin 4, type II} & \scriptsize{protein encoded by this gene is a member of the keratin gene family, associated with White Sponge Nevus, characterized by oral, esophageal, and anal leukoplakia.} \\
					      &  \small{\textit{CRNN}} & \footnotesize{cornulin} & \scriptsize{may play a role in the mucosal/epithelial immune response and epidermal differentiation. } \\
\hline
 \multirow{3}{4em}{\footnotesize{cluster 13, light violet}} &  \small{\textit{PRSS1}} & \footnotesize{protease, serine 1} & \scriptsize{secreted by pancreas, associated with pancreatitis}\\
 					      &  \small{\textit{CPA1}} & \footnotesize{carboxypeptidase A1} & \scriptsize{secreted by pancreas, linked to pancreatitis and pancreatic cancer} \\
					      &  \small{\textit{PNLIP}} & \footnotesize{pancreatic lipase} & \scriptsize{encodes a carboxyl esterase that hydrolyzes insoluble, emulsified triglycerides, and is essential for the efficient digestion of dietary fats. This gene is expressed specifically in the pancreas.}\\
\hline				      
 \multirow{3}{4em}{\footnotesize{cluster 14, coral}} &  \small{\textit{MUC7}} & \footnotesize{mucin 7, secreted} & \scriptsize{encodes a small salivary mucin, thought to play a role in facilitating the clearance of bacteria in the oral cavity and to aid in mastication, speech, and swallowing, associated with susceptibility to asthma.} \\
 					      &  \small{\textit{ALB}} & \footnotesize{albumin} & \scriptsize{functions primarily as a carrier protein for steroids, fatty acids, and thyroid hormones and plays a role in stabilizing extracellular fluid volume.} \\
					      &  \small{\textit{HP}} & \footnotesize{haptoglobin} & \scriptsize{encodes a preproprotein, which subsequently  produces haptoglobin, linked to diabetic nephropathy, Crohn's disease, inflammatory disease behavior and reduced incidence of Plasmodium falciparum malaria.}\\
\hline					      
 \multirow{3}{4em}{\footnotesize{cluster 15, steel blue}}&  \small{\textit{PRL}} & \footnotesize{prolactin 2} & \scriptsize{encodes the anterior pituitary hormone prolactin. This secreted hormone is a growth regulator for many tissues, including cells of the immune system.} \\
 					      &  \small{\textit{GH1}} & \footnotesize{growth hormone 1} & \scriptsize{expressed in the pituitary, play an important role in growth control, mutations in or deletions of the gene lead to growth hormone deficiency and short stature.}\\
					      &  \small{\textit{POMC}} & \footnotesize{proopiomelanocortin} & \scriptsize{synthesized mainly in corticotroph cells of the anterior pituitary, mutations in this gene have been associated with early onset obesity, adrenal insufficiency, and red hair pigmentation.} \\
\hline
 \end{tabular}

\clearpage

