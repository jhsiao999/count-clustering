
\begin{thebibliography}{9}

%\bibitem{Sanders2014} 
%S Anders, T P Pyl, W Huber.
%\textit{HTSeq :  A Python framework to work with high-throughput sequencing data}. 
%Bioinformatics, 2014, in print; online at doi:10.1093/bioinformatics/btu638
%
%\bibitem{Li2009} 
%Li H.*, Handsaker B.*, Wysoker A., Fennell T., Ruan J., Homer N., Marth G., Abecasis G., Durbin R. and 1000 Genome Project Data Processing Subgroup.
%\textit{The Sequence alignment/map (SAM) format and SAMtools}. 
%Bioinformatics, 2009, 25, 2078-9. [PMID: 19505943]
%
%\bibitem{Liao2013}
%Liao Y, Smyth GK and Shi W. 
%\textit{The Subread aligner: fast, accurate and scalable read mapping by seed-and-vote}.
%Nucleic Acids Research, 2013, 41, pp. e108.

%\bibitem{Robinson2010}
%Robinson MD, McCarthy DJ and Smyth GK. 
%\textit{edgeR: a Bioconductor package for differential expression analysis of digital gene expression data}.
%Bioinformatics, 2010, 26, pp. -1.

%\bibitem{Ritchie2015}
%Ritchie ME, Phipson B, Wu D, Hu Y, Law CW, Shi W and Smyth GK. 
%\textit{limma powers differential expression analyses for RNA-sequencing and microarray studies}.
%Nucleic Acids Research, 2015, 43(7), pp. e47.

\bibitem{Eisen1998}
Eisen MB, Spellman PT, Brown PO and Botstein D. 1998.
Cluster analysis and display of genome-wide expression patterns.
\textit{PNAS}, 95(25): 14863-14868

\bibitem{Golub1999}
Golub TR, Slonim DK, Tamayo P, Huard C, Gaasenbeek M, Mesirov JP, Coller H, Loh ML, Downing JR, Caligiuri MA, Bloomfield CD, Lander ES. 1999.
Molecular classification of cancer: class discovery and class prediction by gene expression monitoring.
\textit{Science}, 286(5439): 531-7

\bibitem{Alizadeh2000}
Alizadeh AA1, Eisen MB, Davis RE, Ma C, Lossos IS, Rosenwald A, Boldrick JC \textit{et al}. 2000.
Distinct types of diffuse large B-cell lymphoma identified by gene expression profiling.
\textit{Nature}, 403(6769):  503-11

\bibitem{D'haeseleer2005}
D'haeseleer P. 2005.
How does gene expression clustering work?
\textit{Nat Biotechnol}, 23(12):1499-501

\bibitem{Jiang2004}
Jiang D, Tang C, Zhang A.
Cluster Analysis for Gene Expression Data: A Survey. 
\textit{Microsoft Research}, http://research.microsoft.com/en-us/people/djiang/tkde04.pdf.

\bibitem{Erosheva2006}
Erosheva EA. 2006.
Latent class representation of the grade of membership model.
Seattle: University of Washington.

%\bibitem{Jiang2011}
%Jiang L, Schlesinger F, Davis CA, Zhang Y, Li R, Salit M, Gingeras TR and Oliver B. 
%\textit{Synthetic spike-in standards for RNA-seq experiments}.
%Genome Res, 2011:21, 1543-1551

%\bibitem{Frazee2011}
%Frazee AC, Langmead B, Leek JT. 
%\textit{ReCount: a multi-experiment resource of analysis-ready RNA-seq gene count datasets}. 
%BMC Bioinformatics, 2011, 12:449.

%\bibitem{Zeisel2015}
%Amit Zeisel, Ana B. Mu�oz-Manchado, Simone Codeluppi, Peter L�nnerberg, Gioele La Manno, Anna Jur�us, Sueli Marques, Hermany Munguba, Liqun He, Christer Betsholtz, Charlotte Rolny, Gon�alo Castelo-Branco, Jens Hjerling-Leffler, and Sten Linnarsson.
%\textit{Cell types in the mouse cortex and hippocampus revealed by single-cell RNA-seq}.
%Science 6 March 2015: 347 (6226), 1138-1142.

\bibitem{GTEX2013}
The GTEx Consortium. 2013.
The Genotype-Tissue Expression (GTEx) project. 
\textit{Nature genetics}. 45(6): 580-585. doi:10.1038/ng.2653.

\bibitem{Oshlack2010}
Oshlack A, Robinsom MD, Young MD. 2010.
From RNA-seq reads to differential expression results.
\textit{Genome Biology}. 11:220, DOI: 10.1186/gb-2010-11-12-220

\bibitem{Taddy2012}
Matt Taddy. 2012.
On Estimation and Selection for Topic Models. 
\textit{AISTATS 2012, JMLR W\&CP 22}.
(maptpx R package).

\bibitem{Pritchard2000}
Pritchard, Jonathan K., Matthew Stephens, and Peter Donnelly. 2000.
Inference of population structure using multilocus genotype data. 
\textit{Genetics}. 155.2,  945-959.

\bibitem{Rosenberg2002}
Rosenberg NA, Pritchard JK,  Weber JL, Cann HM,  Kidd KK,  Zhivotovsky LA,  Feldman MW. 2002.
The genetic structure of human populations. 
\textit{Science}. 298,  2381-2385. 

\bibitem{Raj2014}
Raj A,  Stephens M,  Pritchard JK.  2014.
fastSTRUCTURE: Variational Inference of Population Structure in Large SNP Data Sets.
\textit{Genetics}. 197,  573-589.

\bibitem{Falush2003}
Falush D,  Stephens M,  Pritchard JK. 2003.
Inference of population structure using multilocus genotype data: linked loci and correlated allele frequencies.
\textit{Genetics}. 164(4), 1567-87.

\bibitem{Engelhardt2010}
Engelhardt BE,  Stephens M. 2010.
Analysis of Population Structure: A Unifying Framework and Novel Methods Based on Sparse Factor Analysis.
\textit{PLOS Genetics}. DOI: 10.1371/journal.pgen.1001117.

\bibitem{Maaten2008}
van der Maaten LJP and  Hinton GE. 2008.
Visualizing High-Dimensional Data Using t-SNE. 
\textit{J. Mach. Learn. Res.}.  2579-2605.

\bibitem{Maaten2014}
L.J.P. van der Maaten. 2014.
Accelerating t-SNE using Tree-Based Algorithms. 
\textit{J. Mach. Learn. Res.}.  3221-3245.

%\bibitem{Thompson2014}
%Mark A, Thompson R and Wu C. 
%\textit{mygene: Access MyGene.Info services}. 
%2014. R package version 1.2.3.

%\bibitem{Law2014}
%Law CW, Chen Y, Shi W, Smyth GK. 
%\textit{voom: precision weights unlock linear model analysis tools for RNA-seq read counts}. 
%Genome Biology. 2014;15(2):R29. 

\bibitem{Jaitin2014}
Jaitin DA,  Kenigsberg E et al. 2014.
Massively Parallel Single-Cell RNA-Seq for Marker-Free Decomposition of Tissues into Cell Types.
\textit{Science}. 343 (6172) 776-779.

\bibitem{Deng2014}
Deng Q,  Ramskold D,  Reinius B,  Sandberg R. 2014.
Single-Cell RNA-Seq Reveals Dynamic, Random Monoallelic Gene Expression in Mammalian Cells.
\textit{Science}.  343 (6167) 193-196.


\bibitem{Leek2007}
Leek JT,  Storey JD. 2007.
Capturing Heterogeneity in Gene Expression Studies by Surrogate Variable Analysis
\textit{PLoS Genet}. 3(9): e161. doi:10.1371/journal.pgen.0030161

\bibitem{Stegle2012}
Stegle O,  Parts L ,  Piipari M,  Winn J,  Durbin R. 2012.
Using probabilistic estimation of expression residuals (PEER) to obtain increased power and interpretability of gene expression analyses.
\textit{Nat Protoc.}. 7(3):500-7. doi: 10.1038/nprot.2011.457.


\bibitem{Leek2010}
Leek JT,  Scharpf RB,  Bravo HC,  Simcha D,  Langmead B,  Johnson WE,  Geman D,  Baggerly K,  Irizarry RA. 2010.
Tackling the widespread and critical impact of batch effects in high-throughput data.
\textit{Nature Reviews Genetics}. 11, 733-739.

\bibitem{Hicks2015}
Hicks SC, Teng M, Irizarry RA. 2015.
On the widespread and critical impact of systematic bias and batch effects in single-cell RNA-Seq data.
\textit{BiorXiv}. http://biorxiv.org/content/early/2015/09/04/025528

%\bibitem{Houzel2005}
%Herculano-Houzel S and Lent R.
%\textit{Isotropic fractionator: a simple, rapid method for the quantification of total cell and neuron numbers in the brain}.
%J Neurosci. 2005 Mar 9;25(10), 2518-21.

\bibitem{Thompson2014}
Mark A, Thompson R and Wu C.  2014.
mygene: Access MyGene.Info services. 
\textit{R package version 1.2.3.}. 

%
%\bibitem{Gr�n2015}
%Gr�n D, Lyubimova A, Kester L, Wiebrands K, Basak O,  Sasaki N,  Clevers H, van Oudenaarden A.
%\textit{Single-cell messenger RNA sequencing reveals rare intestinal cell types}.
%Nature. 2015 Sep 10;525(7568), 251-5.

%\bibitem{Buettner2015}
%Buettner F, Natarajan KN, Casale FP, Proserpio V, Scialdone A, Theis FJ, Teichmann SA, Marioni JC and Stegle O.
%\textit{Computational analysis of cell-to-cell heterogeneity in single-cell RNA-sequencing data reveals hidden subpopulations of cells}
%Nature Biotechnology 2015, 33, 155?160, doi:10.1038/nbt.3102

%\bibitem{Palmer2005}
%Palmer C, Diehn M, Alizadeh AA and Brown PO.
%\textit{Cell-type specific gene expression profiles of leukocytes in human peripheral blood}.
%BMC Genomics 2006, 7:115.

\bibitem{Flutre2013}
Flutre T,  Wen X,  Pritchard J and Stephens M. 2013.
A Statistical Framework for Joint eQTL Analysis in Multiple Tissues.
\textit{PLoS Genet}. 9(5): e1003486. doi:10.1371/journal.pgen.1003486

\bibitem{Blei2003}
Blei DM,  Ng AY, Jordan MI. 2003.
Latent Dirichlet Allocation.
\textit{J. Mach. Learn. Res.}. 3, 993-1022

\bibitem{Blei2009}
Blei DM, Lafferty J. 2009.
Topic Models.
\textit{In A. Srivastava and M. Sahami, editors, Text Mining: Classification, Clustering, and Applications . Chapman $\&$ Hall/CRC Data Mining and Knowledge Discovery Series}.

\bibitem{Shen-Orr2010}
 Shen-Orr SS,  Tibshirani R,   Khatri, P,  Bodian DL,  Staedtler F,  Perry NM,  Hastie, T,   Sarwal MM,  Davis MM,  Butte AJ. 2010.
 Cell typespecific gene expression   differences   in   complex   tissues.   
 \textit{Nature   Methods}.  7(4),   287-289  
 
\bibitem{Qiao2012}
Qiao W,  Quon G, Csaszar E, Yu  M,  Morris Q,  Zandstra  PW. 2012.   
PERT: A   Method   for   Expression   Deconvolution   of   Human   Blood   Samples   from   Varied
Microenvironmental   and   Developmental   Conditions.   
\textit{PLoS   Comput   Biol}.  8(12)
 
 \bibitem{Repsilber2010}
 Repsilber D,  Kern S,  Telaar A,  Walzl G,  Black GF,   Selbig   J,   Parida  SK,  Kaufmann SH,  Jacobsen M. 2010.
 Biomarker discovery in heterogeneous tissue samples -taking the in-silico deconfounding approach. 
 \textit{BMC bioinformatics}.11(1), 27+
 
 \bibitem{Schwartz2010}
 Schwartz R, Shackney SE. 2010.
 Applying unmixing to gene expression data for tumor phylogeny inference. 
 \textit{BMC bioinformatics}.  11(1), 42+
 
 \bibitem{Lindsay2013}
 Lindsay J,  Mandoiu I, Nelson C. 2013.
 Gene Expression Deconvolution using Single-cells
 \url{http://dna.engr.uconn.edu/bibtexmngr/upload/Lal.13.pdf}.
 
 \bibitem{Hu2016}
 Hu JG,  Shi LL,  Chen YJ,  Xie XM,  Zhang N,  Zhu AY,  Zheng JS,  Feng YF,  Zhang C,  Xi J,  Lu HZ. 2016.
 Differential effects of myelin basic protein-activated Th1 and Th2 cells on the local immune microenvironment of injured spinal cord.
 \textit{Experimental Neurology}. 277, 190-201
 
 \bibitem{duVerle2016}
 duVerle D, Tsuda K. 2016.
 cellTree: Inference and visualisation of Single-Cell RNA-seq data as a hierarchical tree structure. 
 \textit{R package version 1.1.0}, \url{http://tsudalab.org}.
 
 \bibitem{Renard2013}
 Renard M,  Callewaert B,  Baetens M,  Campens L,  MacDermot K et al. 2013.
 Novel MYH11 and ACTA2 mutations reveal a role for enhanced TGF$\beta$ signaling in FTAAD
 \textit{Int J Cardiol}. 165(2), 314-321.
 
 \bibitem{Gong2006}
 Gong B,  Cao Z,  Zheng P,  Vitolo OV,  Liu S,  Staniszewski A,  Moolman D,  Zhang H,  Shelanski M,  Arancio O. 2006.
 Ubiquitin Hydrolase Uch-L1 Rescues $\beta$-Amyloid-Induced Decreases in Synaptic Function and Contextual Memory
 \textit{Cell}. 126(4), 775-788
 
 \bibitem{Hoffman2010}
 Hoffman MD,  Blei DM,  Bach F. 2010. 
 Online learning for latent Dirichlet allocation.
 \textit{Neural Information Processing Systems}.
 
 
    
 
\end{thebibliography}

