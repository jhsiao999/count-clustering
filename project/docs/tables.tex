\clearpage
\begin{table}[htp]
\caption{Cluster Annotations GTEx V6 data} \label{tab:tab1}
\begin{center}
\begin{tabular}{|p{0.7in}|p{0.5in}|p{1.4in}|p{3.6in}|} 
\hline
Cluster & Gene names & Proteins  & Summary \\
\hline

 \multirow{3}{4em}{\small{cluster 1, purple (Nerve, Adipose)} } &  \small{FABP4} & \footnotesize{ fatty acid binding protein 4, adipocyte} & \scriptsize{FABP4 encodes the fatty acid binding protein found in adipocytes, roles include fatty acid uptake, transport, and metabolism}\\ 
 					      & \small{APOD} & \footnotesize{apolipoprotein D} & \scriptsize{encodes a component of high density lipoprotein that has no marked similarity to other apolipoprotein sequences, closely associated with lipoprotein metabolism.}\\
					      & \small{PLIN1} & \footnotesize{perilipin} 1 & \scriptsize{coats lipid storage droplets in adipocytes, thereby protecting them until they can be broken down by hormone-sensitive lipase.} \\
 \hline
 \multirow{3}{4em}{\small{cluster 2, light purple(Arteries, Esophagus)} } & \small{MYH11} & \footnotesize{myosin, heavy chain 11, smooth muscle} & \scriptsize{functions as a major contractile protein, converting chemical energy into mechanical energy through the hydrolysis of ATP.} \\
 					    & \small{ACTA2} & \footnotesize{actin, alpha 2, smooth muscle, aorta} & \scriptsize{protein encoded by this gene belongs to the actin family of proteins, which are highly conserved proteins that play a role in cell motility, structure and integrity, defects in this gene cause aortic aneurysm familial thoracic type 6.}\\
					    &  \small{ACTG2} & \footnotesize{actin, gamma 2, smooth muscle, enteric} & \scriptsize{encodes actin gamma 2; a smooth muscle actin found in enteric tissues, involved in various types of cell motility and in the maintenance of the cytoskeleton. }\\
 \hline
 \multirow{3}{4em}{\small{cluster 3, red (Brain)}} & \small{MBP} & \footnotesize{myelin basic protein} & \scriptsize{major constituent of the myelin sheath of oligodendrocytes and Schwann cells in the nervous system} \\
 					    & \small{GFAP} & \footnotesize{glial fibrillary acidic protein} & \scriptsize{encodes one of the major intermediate filament proteins of mature astrocytes, mutations casuses Alexander disease.} \\
					    & \small{SNAP25}  & \footnotesize{synaptosomal-associated protein, 25kDa} & \scriptsize{this gene product is a presynaptic plasma membrane protein involved in the regulation of neurotransmitter release.}\\
\hline
 \multirow{3}{4em}{\small{cluster 4, light red (Testis)}} & \small{PRM2} & \footnotesize{protamine 2} & \scriptsize{Protamines are the major DNA-binding proteins in the nucleus of sperm} \\
 					      & \small{PRM1} & \footnotesize{protamine 1} & \scriptsize{Protamines are the major DNA-binding proteins in the nucleus of sperm}  \\
					      & \small{PHF7} & \footnotesize{PHD finger protein 7} & \scriptsize{This gene is expressed in the testis in Sertoli cells but not germ cells, regulates spermatogenesis.} \\
\hline					      					      
 \multirow{3}{4em}{\small{cluster 5, blue (Thyroid, Stomach}} & \small{TG} & \footnotesize{thyroglobulin} & \scriptsize{thyroglobulin produced predominantly in thyroid gland, synthesizes thyroxine and triiodothyronine, associated with Graves disease and Hashimotot thyroiditis.} \\
 					     & \small{LIPF} & \footnotesize{lipase, gastric} & \scriptsize{encodes gastric lipase, an enzyme involved in the digestion of dietary triglycerides in the gastrointestinal tract, and responsible for 30 $\%$ of fat digestion processes occurring in human.} \\
					     & \small{PGC} & \footnotesize{progastricsin (pepsinogen C)} & \scriptsize{encodes an aspartic proteinase that belongs to the peptidase family A1. The encoded protein is a digestive enzyme that is produced in the stomach and constitutes a major component of the gastric mucosa,  associated with susceptibility to gastric cancers.}\\					     
 \hline
 \multirow{3}{4em}{\small{cluster 6, sky blue (Skin)}} & \small{KRT10} & \footnotesize{keratin 10, type I} & \scriptsize{encodes a member of the type I (acidic) cytokeratin family, mutations associated with epidermolytic hyperkeratosis.}\\
 					    &  \small{KRT1} & \footnotesize{keratin 1, type II} & \scriptsize{specifically expressed in the spinous and granular layers of the epidermis with family member KRT10 and mutations in these genes have been associated with bullous congenital ichthyosiform erythroderma.} \\
					    & \small{KRT2} & \footnotesize{keratin 2, type II} & \scriptsize{expressed largely in the upper spinous layer of epidermal keratinocytes and mutations in this gene have been associated with bullous congenital ichthyosiform erythroderma.}\\
 \hline
 \multirow{3}{4em}{\small{cluster 7, orange (Cells fibroblasts)}} & \small{FN1}  & \footnotesize{fibronectin 1} & \scriptsize{Fibronectin is involved in cell adhesion, embryogenesis, blood coagulation, host defense, and metastasis.} \\
 					      & \small{COL1A1} & \footnotesize{collagen, type I, alpha 1} & \scriptsize{Mutations in this gene associated with osteogenesis imperfecta types I-IV, Ehlers-Danlos syndrome type and Classical type, Caffey Disease}. \\
					      & \small{COL1A2} & \footnotesize{collagen, type I, alpha 2} & \scriptsize{Mutations in this gene associated with osteogenesis imperfecta types I-IV, Ehlers-Danlos syndrome type and Classical type, Caffey Disease}. \\
\hline
\end{tabular}
\end{center}
\end{table}


%
%
%\begin{table}
%\newpage
%\begin{center}
%\begin{tabular}{|c|p{0.5in}|p{1.9in}|p{2.5in}|} 
%\hline
%Cluster & Gene names & Proteins  & Summary \\
%\hline
%
% \multirow{3}{4em}{cluster black (Testis)} & PRM2 & protamine 2 & \small{Protamines are the major DNA-binding proteins in the nucleus of sperm} \\
% 					      & PRM1 & protamine 1 & \small{Protamines are the major DNA-binding proteins in the nucleus of sperm}  \\
%					      & PHF7 & PHD finger protein 7 & \small{This gene is expressed in the testis in Sertoli cells but not germ cells, regulates spermatogenesis.} \\
%\hline					      					      
% \multirow{3}{4em}{cluster light blue (Thyroid, Stomach} & TG & thyroglobulin & \small{thyroglobulin produced predominantly in thyroid gland, synthesizes thyroxine and triiodothyronine, associated with Graves disease and Hashimotot thyroiditis.} \\
% 					     & LIPF & lipase, gastric & \small{encodes gastric lipase, an enzyme involved in the digestion of dietary triglycerides in the gastrointestinal tract, and responsible for 30 $\%$ of fat digestion processes occurring in human.} \\
%					     & PGC & progastricsin (pepsinogen C) & \small{encodes an aspartic proteinase that belongs to the peptidase family A1. The encoded protein is a digestive enzyme that is produced in the stomach and constitutes a major component of the gastric mucosa,  associated with susceptibility to gastric cancers.}\\					     
% \hline
% \multirow{3}{4em}{cluster deep blue (Skin)} & KRT10 & keratin 10, type I & \small{encodes a member of the type I (acidic) cytokeratin family, mutations associated with epidermolytic hyperkeratosis.}\\
% 					    &  KRT1 & keratin 1, type II & \small{specifically expressed in the spinous and granular layers of the epidermis with family member KRT10 and mutations in these genes have been associated with bullous congenital ichthyosiform erythroderma.} \\
%					    & KRT2 & keratin 2, type II & \small{expressed largely in the upper spinous layer of epidermal keratinocytes and mutations in this gene have been associated with bullous congenital ichthyosiform erythroderma.}\\
% \hline
% \multirow{3}{4em}{cluster dark brown (Cells fibroblasts)} & FN1  & fibronectin 1 & \small{Fibronectin is involved in cell adhesion, embryogenesis, blood coagulation, host defense, and metastasis.} \\
% 					      & COL1A1 & collagen, type I, alpha 1 & \small{Mutations in this gene associated with osteogenesis imperfecta types I-IV, Ehlers-Danlos syndrome type and Classical type, Caffey Disease}. \\
%					      & COL1A2 & collagen, type I, alpha 2 & \small{Mutations in this gene associated with osteogenesis imperfecta types I-IV, Ehlers-Danlos syndrome type and Classical type, Caffey Disease}. \\
%\hline	
% \end{tabular}
% \end{center}
%\end{table}
%




\newpage
\begin{table}
\begin{center}
\begin{tabular}{|p{0.7in} |p{0.5in}|p{1.4in}|p{3.6in}|}
\hline
Cluster & Gene symbol & Gene name  & Summary \\
\hline
\multirow{3}{4em}{\small{cluster 8, light orange (Lung)}} & \small{SFTPB} & \footnotesize{surfactant protein B} & \scriptsize {an amphipathic surfactant protein essential for lung function and homeostasis after birth, muttaions cause pulmonary alveolar proteinosis, fatal respiratory distress in the neonatal period.} \\
 					    & \small{SFTPA2} & \footnotesize{surfactant protein A2} & \scriptsize{Mutations in this gene and a highly similar gene located nearby, which affect the highly conserved carbohydrate recognition domain, are associated with idiopathic pulmonary fibrosis.} \\
					    & \small{SFTPA1} & \footnotesize{surfactant protein A1} &  \scriptsize{encodes a lung surfactant protein that is a member of C-type lectins called collectins, associated with idiopathic pulmonary fibrosis.} \\
\hline				   
 \multirow{3}{4em}{\small{cluster 9, green (Muscle skeletal)}} & \small{MYH1} & \footnotesize{myosin, heavy chain 1, skeletal muscle, adult }& \scriptsize{a major contractile protein which converts chemical energy into mechanical energy through the hydrolysis of ATP.} \\
 					    & \small{NEB} & \footnotesize{nebulin} & \scriptsize{encodes nebulin, a giant protein component of the cytoskeletal matrix that coexists with the thick and thin filaments within the sarcomeres of skeletal muscle, associated with recessive nemaline myopathy.} \\
					    & \small{MYH2} & \footnotesize{myosin, heavy chain 2, skeletal muscle, adult} & \scriptsize{encodes a member of the class II or conventional myosin heavy chains, and functions in skeletal muscle contraction.} \\
\hline					    
 \multirow{3}{4em}{\small{cluster 10, light green (Blood)}} & \small{HBB} & \footnotesize{hemoglobin, beta} & \scriptsize{mutant beta globin causes sickle cell anemia, absence of beta chain/ reduction in beta globin leads to thalassemia.}\\
 					      & \small{HBA2} & \footnotesize{hemoglobin, alpha 2} & \scriptsize{deletion of alpha genes may lead to alpha thalassemia.}  \\
					      & \small{HBA1} & \footnotesize{hemoglobin, alpha 1} & \scriptsize{deletion of alpha genes may lead to alpha thalassemia.}  \\
\hline				      

 \multirow{3}{4em}{\small{cluster 11, mint (Heart)}} &  \small{NPPA} & \footnotesize{natriuretic peptide A} & \scriptsize{protein encoded by this gene belongs to the natriuretic peptide family, associated with atrial fibrillation familial type 6.} \\
 					      &  \small{MYH6} & \footnotesize{myosin, heavy chain 6, cardiac muscle, alpha} & \scriptsize{encodes the alpha heavy chain subunit of cardiac myosin, mutations in this gene cause familial hypertrophic cardiomyopathy and atrial septal defect 3.} \\
					      &  \small{ACTC1} & \footnotesize{actin, alpha, cardiac muscle 1} & \scriptsize{protein encoded by this gene belongs to the actin family, associated with idiopathic dilated cardiomyopathy (IDC) and familial hypertrophic cardiomyopathy (FHC).} \\
\hline			      
 \multirow{3}{4em}{\small{cluster 12, light yellow (Esophagus mucosa)}} &  \small{KRT13} & \footnotesize{keratin 13, type I} & \scriptsize{protein encoded by this gene is a member of the keratin gene family, associated with the autosomal dominant disorder White Sponge Nevus.}\\
 					      &  \small{KRT4} & \footnotesize{keratin 4, type II} & \scriptsize{protein encoded by this gene is a member of the keratin gene family, associated with White Sponge Nevus, characterized by oral, esophageal, and anal leukoplakia.} \\
					      &  \small{CRNN} & \footnotesize{cornulin} & \scriptsize{may play a role in the mucosal/epithelial immune response and epidermal differentiation. } \\
\hline
 \multirow{3}{4em}{\small{cluster 13, light violet (Pancreas)}} &  \small{PRSS1} & \footnotesize{protease, serine 1} & \scriptsize{secreted by pancreas, associated with pancreatitis}\\
 					      &  \small{CPA1} & \footnotesize{carboxypeptidase A1} & \scriptsize{secreted by pancreas, linked to pancreatitis and pancreatic cancer} \\
					      &  \small{PNLIP} & \footnotesize{pancreatic lipase} & \scriptsize{encodes a carboxyl esterase that hydrolyzes insoluble, emulsified triglycerides, and is essential for the efficient digestion of dietary fats. This gene is expressed specifically in the pancreas.}\\
\hline				      
 \multirow{3}{4em}{\small{cluster 14, coral (Liver)}} &  \small{MUC7} & \footnotesize{mucin 7, secreted} & \scriptsize{encodes a small salivary mucin, thought to play a role in facilitating the clearance of bacteria in the oral cavity and to aid in mastication, speech, and swallowing, associated with susceptibility to asthma.} \\
 					      &  \small{ALB} & \footnotesize{albumin} & \scriptsize{functions primarily as a carrier protein for steroids, fatty acids, and thyroid hormones and plays a role in stabilizing extracellular fluid volume.} \\
					      &  \small{HP} & \footnotesize{haptoglobin} & \scriptsize{encodes a preproprotein, which subsequently  produces haptoglobin, linked to diabetic nephropathy, Crohn's disease, inflammatory disease behavior and reduced incidence of Plasmodium falciparum malaria.}\\
\hline					      
 \multirow{3}{4em}{\small{cluster 15, steel blue (Pituitary)}}&  \small{PRL} & \footnotesize{prolactin 2} & \scriptsize{encodes the anterior pituitary hormone prolactin. This secreted hormone is a growth regulator for many tissues, including cells of the immune system.} \\
 					      &  \small{GH1} & \footnotesize{growth hormone 1} & \scriptsize{expressed in the pituitary, play an important role in growth control, mutations in or deletions of the gene lead to growth hormone deficiency and short stature.}\\
					      &  \small{POMC} & \footnotesize{proopiomelanocortin} & \scriptsize{synthesized mainly in corticotroph cells of the anterior pituitary, mutations in this gene have been associated with early onset obesity, adrenal insufficiency, and red hair pigmentation.} \\
\hline
 \end{tabular}
 \end{center}
\end{table}

\clearpage

\subsection{Supplementary Table 1}

\begin{table}[htp]
\begin{center}
\caption{Cluster Annotations GTEx V6 Brain data} \label{tab:tab2}
\begin{tabular}{|p{0.7in}|p{0.5in}|p{1.4in}|p{3.6in}|} 
\hline
Cluster & Gene symbol & Gene name & Summary \\
\hline
 \multirow{3}{4em}{\small{cluster 1, royal blue}}  &  \small{ATP1A2} & \footnotesize{ ATPase, Na+/K+ transporting, alpha 2 polypeptide} & \scriptsize{responsible for establishing and maintaining the electrochemical gradients of Na and K ions across the plasma membrane, mutations in this gene result in familial basilar or hemiplegic migraines, and in a rare syndrome known as alternating hemiplegia of childhood.}   \\ 
 					      & \small{CLU} &  \footnotesize{clusterin} & \scriptsize{protein encoded by this gene is a secreted chaperone that can under some stress conditions also be found in the cell cytosol, also involved in cell death, tumor progression, and neurodegenerative disorders.} \\
					      & \small{DNAJB1} & \footnotesize{DnaJ (Hsp40) homolog, subfamily B, member 1} & \scriptsize{encodes a member of the DnaJ or Hsp40 (heat shock protein 40 kD) family of proteins, that stimulates the ATPase activity of Hsp70 heat-shock proteins  to promote protein folding and prevent misfolded protein aggregation.} \\
\hline
 \multirow{3}{4em}{\small{cluster 2, yellow orange}} & \small{SNAP25} & \footnotesize{synaptosomal-associated protein, 25kDa} & \scriptsize{Synaptic vesicle membrane docking and fusion is mediated by SNAREs located on the vesicle membrane (v-SNAREs) and the target membrane (t-SNAREs), involved in the regulation of neurotransmitter release.}\\
 					    & \small{ENO2} & \footnotesize{enolase 2 (gamma, neuronal)} & \scriptsize{ encodes one of the three enolase isoenzymes found in mammals, is found in mature neurons and cells of neuronal origin.} \\
					    &  \small{CHGB} &  \footnotesize{chromogranin B} & \scriptsize{ encodes a tyrosine-sulfated secretory protein abundant in peptidergic endocrine cells and neurons. This protein may serve as a precursor for regulatory peptides.} \\
 \hline
 
 \multirow{3}{4em}{\small{cluster 3, turquoise}}  &  \small{CALM3} & \footnotesize{calmodulin 3 (phosphorylase kinase, delta)} & \scriptsize{ is a calcium binding protein that plays a role in signaling pathways, cell cycle progression and proliferation.}   \\ 
 					      & \small{FBXL16} &  \footnotesize{F-box and leucine-rich repeat protein 16} & \scriptsize{ Members of the F-box protein family, such as FBXL16, are characterized by an approximately 40-amino acid F-box motif.} \\
					      & \small{UCHL1} & \footnotesize{ubiquitin carboxyl-terminal esterase L1} & \scriptsize{specifically expressed in the neurons and in cells of the diffuse neuroendocrine system. Mutations in this gene may be associated with Parkinson disease.} \\
 \hline
 \multirow{3}{4em}{\small{cluster 4,  red}} & \small{MBP} & \footnotesize{myelin basic protein} & \scriptsize{protein encoded is a major constituent of the myelin sheath of oligodendrocytes and Schwann cells in the nervous system.} \\
 					    & \small{MYH11} & \footnotesize{glial fibrillary acidic protein} & \scriptsize{ encodes major intermediate filament proteins of mature astrocytes, a marker to distinguish astrocytes during development, mutations in this gene cause Alexander disease, a rare disorder of astrocytes in central nervous system.} \\
					    & \small{ACTA2}  & \footnotesize{secreted protein, acidic, cysteine-rich (osteonectin)}  & \scriptsize{encodes a cysteine-rich acidic matrix-associated protein, required for the collagen in bone to become calcified, in extracellular matrix synthesis and cell shape promotion, associated with tumor suppression.}\\
\hline
\end{tabular}
 \end{center} \label{tab:tab2}
\end{table}

\subsection{Supplementary Table 2}
\begin{table}[htp]
\begin{center}
\caption{Cluster Annotations Deng et al (2014) data} \label{tab:tab3}
\begin{tabular}{|c|p{1in}|p{2.3in}|p{2.5 in}|}  
\hline
Cluster & Gene symbol &  Gene name/function & GO Annotations \\
\hline
\multirow{3}{4em}{\small{cluster 1, blue}} & \footnotesize{Bcl2l10} & \footnotesize{Bcl2 like 10} & \multirow{6}{16em} {\footnotesize{GO:0007276 (gamete generation), GO:0032504 (multicellular organism reproduction), GO:0044702 (single organism reproduction), GO:0048477 (oogenesis), GO:0048599 (oocyte development), GO:0009994 (oocyte differentiation), GO:0051321 (meiotic cell cycle), GO:0006306 (DNA methylation), GO:0051302 (regulation of cell division)}}\\ 			 								& \footnotesize{Tcl1} & \footnotesize{T cell lymphoma breakpoint 1} & \\
					    & \footnotesize{E330034G19Rik}  & \footnotesize{RIKEN cDNA E330034G19 gene}  & \\
					    & \footnotesize{LOC100502936} & NA & \\
					    & \footnotesize{Oas1d} & \footnotesize{$2^{'}$-$5^{'}$ oligoadenylate synthetase 1D} & \\
					    & \footnotesize{AU022751} & \footnotesize{expressed sequence AU022751} & \\
					    & \footnotesize{Spin1} & \footnotesize{spindlin 1} & \\
					    & \footnotesize{Khdc1b} & \footnotesize{KH domain containing 1B} & \\
					    & \footnotesize{D6Ertd527e} & \footnotesize{DNA segment, Chr 6, ERATO Doi 527, expressed} &\\
					    & \footnotesize{Btg4} & \footnotesize{B cell translocation gene 4} &\\
\hline
 \multirow{3}{4em}{\small{cluster 2, magenta}} & \footnotesize{Obox3} & \footnotesize{oocyte specific homeobox 3} & \multirow{6}{16em}{\footnotesize{GO:0016604 (nuclear body), GO:0005814 (centriole), GO:0044450 (microtubule organizing center part)}} \\ 					     				& \footnotesize{Zfp352}  & \footnotesize{zinc finger protein 352}  & \\	
 			& \footnotesize{Gm8300} & \footnotesize{predicted gene 8300} & \\
			& \footnotesize{Usp17l5} & \footnotesize{NA} & \\
			& \footnotesize{BB287469} & \footnotesize{	expressed sequence BB287469} & \\
			& \footnotesize{Rfpl4b } & \footnotesize{ ret finger protein-like 4B} & \\
			& \footnotesize{Gm2022} & \footnotesize{predicted pseudogene 2022} & \\
			& \footnotesize{Gm5662} & \footnotesize{predicted gene 5662} & \\
			& \footnotesize{Gm11544 } & \footnotesize{predicted gene 11544} & \\
			& \footnotesize{Gm4850} & \footnotesize{THO complex 4 pseudogene} &\\
\hline
\multirow{3}{4em}{\small{cluster 3, yellow}} & \footnotesize{Rtn2} & \footnotesize{reticulon 2 (Z-band associated protein)} & \multirow{6}{16em} { \footnotesize{GO:0044428 (nuclear part), GO:0031981 (nuclear lumen), GO:0070013 (intracellular organelle lumen), GO:0005730 (nucleolus), GO:0005654 (nucleoplasm),  GO:0003723 (RNA binding), GO:0005874 (microtubule), GO:0043229 (intracellular organelle)}}\\ 
 					& \footnotesize{Ebna1bp2} & \footnotesize{EBNA1 binding protein 2} & \\
					& \footnotesize{Zfp259} & \footnotesize{NA} &\\
					& \footnotesize{Nasp} & \footnotesize{nuclear autoantigenic sperm protein (histone-binding)} & \\
					& \footnotesize{Cenpe} & \footnotesize{centromere protein E} & \\
					& \footnotesize{Rnf216} & \footnotesize{ring finger protein 216} & \\
					& \footnotesize{Ctsl} & \footnotesize{cathepsin L} &  \\
					& \footnotesize{Tor1b} & \footnotesize{torsin family 1, member B} & \\
					& \footnotesize{Ankrd10} & \footnotesize{ankyrin repeat domain 10} & \\
					& \footnotesize{Lamp2} & \footnotesize{lysosomal-associated membrane protein 2} & \\
\hline
 \end{tabular}
 \end{center} \label{tab:tab3}
\end{table}

\begin{table}[htp]
\begin{center}
\caption{Cluster Annotations Deng et al (2014) data} \label{tab:tab4}
\begin{tabular}{|c|p{1in}|p{2.3in}|p{2.5 in}|} 
\hline
Cluster & Gene symbol &  Gene name/function  & GO Annotations\\
\hline
  \multirow{3}{4em}{\small{cluster 4, green}}  &  \footnotesize{Timd2} & \footnotesize{ T cell immunoglobulin and mucin domain containing 2} & \multirow{6}{16em} {\footnotesize{GO:0005829 (cytosol), GO:0044444 (cytoplasmic part), GO:1901575 (organic substance catabolic process), GO:0000151 (ubiquitin ligase com- plex),  GO:0009056 (catabolic process), GO:0072655 (protein localization mitochondrion), GO:0044265 (cellular macromolecule catabolic process), GO:0051082 (unfolded protein binding), GO:0023026 (MHC class II protein complex binding),
 GO:0055131 (C3HC4-type RING finger domain binding)}} \\ 
 					      & \footnotesize{Isyna1} &  \footnotesize{myo-inositol 1-phosphate synthase A1} & \\
					      & \footnotesize{Alppl2} & \footnotesize{alkaline phosphatase, placental-like 2} & \\
					      & \footnotesize{Prame15} & \footnotesize{preferentially expressed antigen in melanoma like 5} & \\
					      & \footnotesize{Hsp90ab1} & \footnotesize{heat shock protein 90 alpha (cytosolic), class B member 1} & \\
					      & \footnotesize{Fbxo15} & \footnotesize{F-box protein 15} & \\
					      & \footnotesize{Tceb1} & \footnotesize{transcription elongation factor B (SIII), polypeptide 1} & \\
					      & \footnotesize{Gpd1l } & \footnotesize{glycerol-3-phosphate dehydrogenase 1-like}  & \\
					      & \footnotesize{Pemt} &\footnotesize{phosphatidylethanolamine  \; N-methyltransferase} & \\
					      & \footnotesize{Hsp90aa1} & \footnotesize{heat shock protein 90, alpha (cytosolic), class A member 1} & \\ 
 \hline
  \multirow{3}{4em}{\small{cluster 5,  purple}} & \footnotesize{Upp1} & \footnotesize{uridine phosphorylase 1} & \multirow{6}{16em} {\footnotesize{GO:0070062 (extracellular exosome), GO:0043230 (extracellular organelle), GO:1903561 (extracellular vesicle), GO:0006950 (response to stress), GO:0006979 (response to oxidative stress), GO:0044710 (metabolic process), GO:0048514 (blood vessel morphogenesis), GO:0001944 (vasculature development), GO:0030198 (extracellular matrix organization)}} \\ 					    
 			    & \footnotesize{Tdgf1} & \footnotesize{teratocarcinoma-derived growth factor 1} & \\
			    &  \footnotesize{Aqp8} &  \footnotesize{aquaporin 8} & \\
			    &  \footnotesize{Fabp5} & \footnotesize{fatty acid binding protein 5, epidermal, protects against atherosclerosis, diet-induced obesity, insulin resistance and experimental autoimmune encephalomyelitis}  & \\
			    &  \footnotesize{Tat} & \footnotesize{tyrosine aminotransferase, regulated by glucocorticoid and polypeptide hormones} & \\
			    & \footnotesize{Pdgfra} & \footnotesize{platelet derived growth factor receptor, alpha polypeptide} & \\
			    & \footnotesize{Pyy } & \footnotesize{peptide YY } & \\
			    & \footnotesize{Prdx1} & \footnotesize{peroxiredoxin 1} & \\
			    & \footnotesize{Col4a1} & \footnotesize{collagen, type IV, alpha 1.} & \\
			    & \footnotesize{Spp1} & \footnotesize{secreted phosphoprotein 1} & \\
 \hline
  \multirow{3}{4em}{\small{cluster 6, orange}}  &  \footnotesize{Actb} & \footnotesize{actin, beta,  involved in cell motility, structure, and integrity}   & \multirow{6}{16em} {\footnotesize{GO:0065010 (extracellular membrane-bounded organelle), GO:0070062 (extracellular exosome),  GO:0043230 (extracellular organelle), GO:1903561 (extracellular vesicle), GO:0031982 (vesicle), GO:0048468 (cell development), GO:0030036 (actin cytoskeleton and organization), GO:0032432 (actin filament bundle),  GO:0005912 (adherens junction)}}\\ 
 					      & \footnotesize{Krt18} &  \footnotesize{keratin 18}  & \\
					      & \footnotesize{Fabp3} & \footnotesize{fatty acid binding protein 3, muscle and heart}  & \\
					      & \footnotesize{Id2} & \footnotesize{inhibitor of DNA binding 2}  & \\
					      & \footnotesize{Tspan8} & \footnotesize{tetraspanin 8} & \\
					      & \footnotesize{Gm2a} & \footnotesize{GM2 ganglioside activator protein} & \\
					      & \footnotesize{Lgals1} & \footnotesize{lectin, galactose binding, soluble 1}  & \\
					      & \footnotesize{Adh1} & \footnotesize{alcohol dehydrogenase 1 (class I) } & \\
					      & \footnotesize{Lrp2} & \footnotesize{low density lipoprotein receptor-related protein 2} & \\
					      & \footnotesize{BC051665} & \footnotesize{cDNA sequence BC051665} & \\
\hline
\end{tabular}
 \end{center} \label{tab:tab4}
\end{table}

\clearpage

\subsection{Supplementary Table 3}

\begin{table}[htp]
\caption{Cluster Annotations GTEx V6 data (with GO annotations): clusters 1-5} \label{tab:tab11}
\begin{center}
\begin{tabular}{|p{0.7in}|p{0.5in}|p{1.4in}|p{3.5in}|} 
\hline
Cluster & Gene names & Proteins  & Summary \\
\hline
 \multirow{3}{4em}{\small{cluster 1, purple (Nerve, Adipose)} } &  \small{FABP4} & \footnotesize{ fatty acid binding protein 4, adipocyte} & 
 \multirow{6}{18em}{\footnotesize{GO:0005578 (proteinaceous extracellular matrix), GO:0005615 (extracellular space), GO:1901700 (response to oxygen-containing compound), GO:0005811 (lipid particle), GO:0009719 (response to endogenous stimulus), GO:0009611 (response to wounding)}} \\
 			& \small{APOD} & \footnotesize{apoliprotein D} & \\
			& \small{PLIN1} & \footnotesize{perilipin 1} & \\
			& \small{MPZ} & \footnotesize{myelin protein zero} & \\
			& \small{GPX3} & \footnotesize{glutathione peroxidase 3} & \\ \hline
 \multirow{3}{4em}{\small{cluster 2, light purple(Arteries, Esophagus)} } & \small{MYH11} &  \footnotesize{myosin, heavy chain 11, smooth muscle} & \multirow{6}{18em}{\footnotesize{GO:0005925 (focal adhesion), GO:0005924 (cell-substrate adherens junction), GO:0015629 (actin cytoskeleton), GO:0001725 (stress fiber), GO:0006936 (muscle contraction), GO:0032432 (actin filament bundle)}} \\
 			& \small{ACTA2} & \footnotesize{actin, alpha 2, smooth muscle, aorta}  &\\
			& \small{ACTG2} & \footnotesize{actin, gamma 2, smooth muscle, enteric} &\\
			& \small{TAGLN} & \footnotesize{transgelin} &\\
			& \small{MYL9} & \footnotesize{myosin light chain} &\\
\hline
\multirow{3}{4em}{\small{cluster 3, red (Brain)}} & \small{MBP} & \footnotesize{myelin basic protein} &	 \multirow{6}{18em}{\footnotesize{GO:0097458 (neuron part), GO:0007268 (synaptic transmission), GO:0007267 (cell-cell signalling), GO:0007399 (nervous system development), GO:0043005 (neuron projection), GO:0036477 (somatodendritic component), GO:0022008 (neurogenesis), GO:0030424 (axon)}} \\
			& \small{GFAP} & \footnotesize{glial fibrillary acidic protein} & \\
			& \small{MTURN} & \footnotesize{maturin, neural progenitor} & \\

			& \small{SNAP25} & \footnotesize{synaptosome associated protein 25kDa} & \\
			& \small{MT3} & \footnotesize{metallothionein 3} & \\
\hline
\multirow{3}{4em}{\small{cluster 4, light red (Testis)}} & \small{PRM2} & \footnotesize{protamine 2} & \multirow{6}{18em}{\footnotesize{GO:0019953 (sexual reproduction), GO:0007276 (gamete generation), GO:0007283 (spermatogenesis), GO:0097228 (sperm principal piece), GO:0035686 (sperm fibrous sheath)}} \\
			& \small{PRM1} & \footnotesize{protamine 1} & \\
			& \small{PHF7} & \footnotesize{PHD finger protein 7} & \\
			& \small{TSACC} & \footnotesize{TSSK6 activating co-chaperone} & \\
			& \small{TEX40} & \footnotesize{testis expressed 40} & \\
\hline
 \multirow{3}{4em}{\small{cluster 5, blue (Thyroid, Stomach}} & \small{TG} & \footnotesize{thyroglobulin} & \multirow{6}{18em}{\footnotesize{GO:0042446 (hormone biosynthetic process), GO:0042445 (hormone metabolic process), GO:0008202 (steroid metabolic process), GO:0006694 (steroid biosynthetic process), GO:0006629 (lipid metabolic process), GO:0065008 (regulation of biological quality)}} \\
 			& \small{LIPF} & \footnotesize{lipase F, gastric type} & \\
			& \small{PGC} & \footnotesize{progastricsin (pepsinogen C)} & \\
			& \small{PGA3} & \footnotesize{pepsinogen 3, group I (pepsinogen A)} & \\
			& \small{CYP11B1} & \footnotesize{cytochrome P450 family 11 subfamily B member 1} & \\
\hline
\end{tabular}
\end{center} \label{tab:tab11}
  \end{table}	
  
  
 \begin{table}[htp]
\caption{Cluster Annotations GTEx V6 data (with GO annotations): clusters 6-10} \label{tab:tab11}
\begin{center}
\begin{tabular}{|p{0.7in}|p{0.5in}|p{1.4in}|p{3.5in}|} 
\hline
Cluster & Gene names & Proteins  & Summary \\
\hline
\multirow{3}{4em}{\small{cluster 6, sky blue (Skin)}} & \small{KRT10} & \footnotesize{keratin 10, type I} & \multirow{6}{18em}{\footnotesize{GO:0043588 (skin development), GO:0008544 (epidermis development), GO:0009888 (tissue development), GO:0060429 (epithelium development), GO:0042633 (hair cycle), GO:0042303 (molting cycle)}} \\
 			& \small{KRT1} & \footnotesize{keratin 1, type II} & \\
			& \small{KRT2} & \footnotesize{keratin 2, type II} & \\
			& \small{LOR} & \footnotesize{loricrin} & \\
			& \small{KRT14} & \footnotesize{keratin 14, type I} & \\
\hline
\multirow{3}{4em}{\small{cluster 7, orange (Cells fibroblasts)}} & \small{FN1}  & \footnotesize{fibronectin 1} & \multirow{6}{18em}{\footnotesize{GO:0030198 (extracellular matrix organization), GO:0005578 (proteinaceous extracellular matrix), GO:0032963 (collagen metabolic process), GO:0005615 (extracellular space), GO:0030574 (collagen catabolic process)}} \\
			& \small{COL1A1} & \footnotesize{collagen type I alpha 1} & \\
			& \small{COL1A2} & \footnotesize{collagen type I alpha 2} & \\
			& \small{COL3A1} & \footnotesize{collagen type III alpha 1} & \\
			& \small{COL6A3} & \footnotesize{collagen type VI alpha 3} & \\
\hline
\multirow{3}{4em}{\small{cluster 8, light orange (Lung)}} & \small{SFTPB} & \footnotesize{surfactant protein B} &  \multirow{6}{18em}{\footnotesize{GO:0002684 (positive regulation of immune system process), GO:0006955 (immune response), GO:0006952 (defense response), GO:0006959 (humoral immune response), GO:0002443 (leukocyte mediated immunity)}} \\
				& \small{SFTPA2} & \footnotesize{surfactant protein A2} & \\
				& \small{SFTPA1} & \footnotesize{surfactant protein A1} & \\
				& \small{SFTPC} & \footnotesize{surfactant protein C} & \\
				& \small{IGHG1} & \footnotesize{immunoglobulin heavy constant gamma 1 (G1m marker)} & \\
\hline
\multirow{3}{4em}{\small{cluster 9, green (Muscle skeletal)}} & \small{MYH1} & \footnotesize{myosin, heavy chain 1, skeletal muscle, adult }&  \multirow{6}{18em}{\footnotesize{GO:0043292 (contractile fiber), GO:0030016 (myofibril), GO:0030017 (sarcomere), GO:0006936 (muscle contraction), GO:0003012 (muscle system process), GO:0015629 (actin cytoskeleton), GO:0008092 (cytoskeletal protein binding)}} \\
			& \small{NEB} & \footnotesize{nebulin} & \\
			& \small{MYH2} & \footnotesize{myosin, heavy chain 2, skeletal muscle, adult} & \\
			& \small{MYBPC1} & \footnotesize{myosin binding protein C, slow type} & \\
			& \small{ACTA1} & \footnotesize{actin, alpha 1, skeletal muscle} & \\
\hline
\multirow{3}{4em}{\small{cluster 10, light green (Blood)}} & \small{HBB} & \footnotesize{hemoglobin, beta} & \multirow{6}{18em}{\footnotesize{GO:0006955 (immune response), GO:0071944 (cell periphery), GO:0005886 (plasma membrane), GO:0005833 (hemoglobin complex), GO:0005344 (oxygen transporter activity), GO:0015669 (gas transport), GO:0009611 (response to wounding)}} \\
 				& \small{HBA2} & \footnotesize{hemoglobin subunit alpha 2} & \\
				& \small{HBA1} & \footnotesize{hemoglobin subunit alpha 1} & \\
				& \small{CSF3R} & \footnotesize{colony stimulating factor 3 receptor} & \\
				& \small{IFITM2} & \footnotesize{interferon induced transmembrane protein 2} & \\
\hline
\end{tabular}
\end{center} \label{tab:tab11}
  \end{table}		
	
			
			
\begin{table}[htp]
\caption{Cluster Annotations GTEx V6 data (with GO annotations): clusters 11-15} \label{tab:tab11}
\begin{center}
\begin{tabular}{|p{0.7in}|p{0.5in}|p{1.4in}|p{3.5in}|} 
\hline
Cluster & Gene names & Proteins  & Summary \\
\hline
 \multirow{3}{4em}{\small{cluster 11, mint (Heart)}} &  \small{NPPA} & \footnotesize{natriuretic peptide A} & \multirow{6}{18em}{\footnotesize{GO:0022904 (espiratory electron transport chain), GO:0030017 (sarcomere), GO:0044449 (contractile fiber part), GO:0030016 (myofibril), GO:0045333 (cellular respiration), GO:0003015 (heart process), GO:0006091 (generation of precursor metabolites and energy), GO:0006936 (muscle contraction)}} \\
 				& \small{MYH6} & \footnotesize{myosin, heavy chain 6, cardiac muscle, alpha} & \\
				& \small{ACTC1} & \footnotesize{actin, alpha, cardiac muscle 1} & \\
				& \small{TNNT2} & \footnotesize{troponin T2, cardiac type} & \\
				& \small{MYBPC3} & \footnotesize{myosin binding protein C, cardiac} & \\
\hline
 \multirow{3}{4em}{\small{cluster 12, light yellow (Esophagus mucosa)}} &  \small{KRT13} & \footnotesize{keratin 13, type I} & \multirow{6}{18em}{\footnotesize{GO:0008544 (epidermis development), GO:0031424 (keratinization), GO:0030855 (epithelial cell differentiation), GO:0065010 (extracellular membrane-bounded organelle), GO:0070062 (extracellular exosome), GO:1903561 (extracellular vesicle)}} \\
 				& \small{KRT4} & \footnotesize{keratin 4, type II} & \\
				& \small{SPRR3} & \footnotesize{small proline rich protein 3} & \\
				& \small{CRNN} & \footnotesize{cornulin} & \\
				& \small{RHCG} & \footnotesize{Rh family C glycoprotein} & \\
\hline
 \multirow{3}{4em}{\small{cluster 13, light violet (Pancreas)}} &  \small{PRSS1} & \footnotesize{protease, serine 1} &  \multirow{6}{18em}{\footnotesize{GO:0007586 (digestion), GO:0004252 (serine-type endopeptidase activity), GO:0008233 (peptidase activity), GO:0044241 (lipid digestion), GO:0006508 (proteolysis), GO:0016787 (hydrolase activity), GO:0016160 (amylase activity)}} \\
 				& \small{PNLIP} & \footnotesize{pancreatic lipase} & \\
				& \small{CPA1} & \footnotesize{carboxypeptidase A1} & \\
				& \small{CELA3A} & \footnotesize{chymotrypsin like elastase family member 3A} & \\
				& \small{GP2} & \footnotesize{glycoprotein 2} \\
\hline
 \multirow{3}{4em}{\small{cluster 14, coral (Liver)}} &  \small{MUC7} & \footnotesize{mucin 7, secreted} & \multirow{6}{18em}{\footnotesize{GO:0005615 (extracellular space), GO:0072562 (blood microparticle), GO:0065010 (extracellular membrane bound organelle), GO:0070062 (extracellular exosome), GO:0002526 (acute inflammatory response), GO:0031982 (vesicle)}} \\
 				& \small{ALB} & \footnotesize{albumin} & \\
				& \small{HP} & \footnotesize{haptoglobin} & \\
				& \small{FGA} & \footnotesize{fibrinogen alpha chain} & \\
				& \small{FGB} & \footnotesize{fibrinogen beta chain} & \\
\hline
 \multirow{3}{4em}{\small{cluster 15, steel blue (Pituitary)}}&  \small{PRL} & \footnotesize{prolactin 2}  & \multirow{6}{16em}{\footnotesize{GO:0005179 (hormone activity), GO:0005148 (prolactin receptor binding), GO:0012505 (endomembrane system), GO:0016486 (peptide hormone processing), GO:0046879 (hormone secretion), GO:0009914 (hormone transport), GO:0050432 (catecholamine secretion)}} \\
 			& \small{GH1} & \footnotesize{growth hormone 1} & \\
			& \small{POMC} & \footnotesize{proopiomelanocortin} & \\
			& \small{CGA} & \footnotesize{glycoprotein hormones, alpha polypeptide} & \\
			& \small{CHGB} & \footnotesize{chromogranin B} & \\
\hline
\end{tabular}
\end{center} \label{tab:tab11}
\end{table}




\clearpage
\subsection{Supplementary Table 4}
\begin{table}[htp]
\begin{center}
\caption{Deng et al (2014) Cluster 1, blue annotation} \label{tab:tab9}
\begin{tabular}{|c|c|p{1.5in}|p{4in}|}
  \hline
 & go.id & name & significant \\ 
  \hline
1 & GO:0007276 & gamete generation & \footnotesize{BCL2L10; GDF9; NOBOX; PABPC1L; RGS2; CREB3L4; RNF114; BMP15; PTTG1; TDRD12; WEE2; SPIN1; DAZL} \\ 
  2 & GO:0007292 & female gamete generation & \footnotesize{GDF9; BCL2L10; PABPC1L; BMP15; WEE2; DAZL; NOBOX} \\ 
  3 & GO:0048609 & multicellular organismal reproductive process & \footnotesize{GDF9; NOBOX; PABPC1L; BCL2L10; BMP15; CREB3L4; TGFB2; RNF114; RGS2; PTTG1; TDRD12; WEE2; SPIN1; DAZL} \\ 
  4 & GO:0032504 & multicellular organism reproduction & \footnotesize{GDF9; NOBOX; PABPC1L; BCL2L10; BMP15; CREB3L4; TGFB2; RNF114; RGS2; PTTG1; TDRD12; WEE2; SPIN1; DAZL}\\ 
  5 & GO:0019953 & sexual reproduction & \footnotesize{BCL2L10; GDF9; NOBOX; PABPC1L; RGS2; CREB3L4; RNF114; BMP15; PTTG1; TDRD12; WEE2; SPIN1; DAZL} \\ 
  6 & GO:0044702 & single organism reproductive process & \footnotesize{GDF9; NOBOX; PABPC1L; BCL2L10; BMP15; CREB3L4; TGFB2; CASP8; RNF114; RGS2; PTTG1; TDRD12; WEE2; SPIN1; DAZL} \\ 
  7 & GO:0048477 & oogenesis & \footnotesize{WEE2; GDF9; NOBOX; PABPC1L; DAZL} \\ 
  8 & GO:0044703 & multi-organism reproductive process & \footnotesize{BCL2L10; GDF9; NOBOX; PABPC1L; RGS2; CREB3L4; RNF114; BMP15; PTTG1; TDRD12; WEE2; SPIN1; DAZL} \\ 
  9 & GO:0048599 & oocyte development  & \footnotesize{WEE2; GDF9; PABPC1L; DAZL} \\ 
  10 & GO:0009994 & oocyte differentiation & \footnotesize{WEE2; GDF9; PABPC1L; DAZL} \\ 
  11 & GO:0051321 & meiotic cell cycle & \footnotesize{H1FOO; WEE2; TDRD12; SPIN1; PTTG1; DAZL} \\ 
  12 & GO:0001556 & oocyte maturation & \footnotesize{WEE2; PABPC1L; DAZL} \\ 
  13 & GO:0006306 & DNA methylation & \footnotesize{TDRD12; H1FOO; TET3; ZFP57} \\ 
  14 & GO:0051302 & regulation of cell division & \footnotesize{TGFB2; PTTG1; TXNIP; WEE2; CHEK1; DAZL} \\ 
  15 & GO:0060255 & regulation of macromolecule metabolic process & \footnotesize{TGFB2; NOBOX; BPGM; UBE2D3; NFYA; CASP8; BMP15; TXNIP; TDRD12; GDF9; BCL2L10} \\ 
 \hline
\end{tabular}
\end{center} \label{tab:tab9}
  \end{table}

\begin{table}[htp]
\begin{center}
\caption{Deng et al (2014) Cluster 2, magenta annotation} \label{tab:tab10}
\begin{tabular}{|c|c|p{1.5in}|p{4in}|}
  \hline
 & go.id & name  & significant \\ 
  \hline
1 & GO:0016604 & nuclear body  & \footnotesize{YTHDC1; RBM8A; CDK12; PSME4; PPP1R8; HIPK1; TOPORS} \\ 
  2 & GO:0005814 & centriole  & \footnotesize{SFI1; PLK2; ROCK1; TOPORS} \\ 
  3 & GO:0044450 & microtubule organizing center part  & \footnotesize{SFI1; PLK2; ROCK1; TOPORS} \\ 
   \hline
\end{tabular}
 \end{center} \label{tab:tab10}
  \end{table}


\begin{table}[htp]
\begin{center}
\caption{Deng et al (2014) Cluster 3, yellow annotation} \label{tab:tab8}
\begin{tabular}{|c|c|p{1.5in}|p{4in}|}
  \hline
 & go.id & name  & significant \\ 
  \hline
1 & GO:0044428 & nuclear part  & \footnotesize{MAD2L2; SMARCC1; PPRC1; SLU7; NFYB; TOR1B; MIOS; NR1H3; POLR3K} \\ 
  2 & GO:0031981 & nuclear lumen & \footnotesize{MAD2L2; SMARCC1; PPRC1; SLU7; NFYB; POLR1E; MIOS; POLR3K; XPO1}\\ 
  3 & GO:0070013 & intracellular organelle lumen & \footnotesize{MAD2L2; SMARCC1; PPRC1; SLU7; NFYB; POLR1E; MIOS; POLR3K; XPO1; DNTTIP2; ZBTB10; ZBTB17} \\ 
  4 & GO:0043233 & organelle lumen & \footnotesize{MAD2L2; SMARCC1; PPRC1; SLU7; NFYB; POLR1E; MIOS; POLR3K; XPO1} \\ 
  5 & GO:0005730 & nucleolus & \footnotesize{XPO1; DNTTIP2; ESF1; WDR43; ZDHHC7; HEATR1; POLR1E; DDX24; POLR3K} \\ 
  6 & GO:0005634 & nucleus & \footnotesize{MAD2L2; SMARCC1; PPRC1; SLU7; NFYB; TOR1B; MIOS; NR1H3; EIF5B; POLR3K} \\ 
  7 & GO:0044446 & intracellular organelle part & \footnotesize{MAD2L2; PTDSS2; SMARCC1; KLHL21; TOR1B; PPRC1; SLU7; NFYB; SLC25A36; ECE2} \\ 
  8 & GO:0005654 & nucleoplasm & \footnotesize{MAD2L2; SMARCC1; PPRC1; SLU7; NFYB; POLR1E; MIOS; POLR3K; XPO1; ZBTB10; ZBTB17} \\ 
  9 & GO:0003723 & RNA binding & \footnotesize{PPRC1; EIF5B; XPO1; DNTTIP2; WDR43; DDX10; EIF3C; BCLAF1; EBNA1BP2; RARS}\\ 
  10 & GO:0003676 & nucleic acid binding & \footnotesize{SMARCC1; PPRC1; SLU7; NFYB; POLR1E; EIF5B; POLR3K; XPO1; DNTTIP2} \\ 
  11 & GO:0043231 & intracellular membrane-bounded organelle &  \footnotesize{MAD2L2; PTDSS2; SMARCC1; TOR1B; PPRC1; SLU7; NFYB; ESF1; ECE2; LMAN1L} \\ 
  12 & GO:0043229 & intracellular organelle & \footnotesize{MAD2L2; PTDSS2; SMARCC1; KLHL21; TOR1B; PPRC1; ARRDC1; SLU7; NFYB; ESF1; ECE2} \\ 
  13 & GO:0005874 & microtubule & \footnotesize{WDR43; KLHL21; HAUS6; CENPE; TEKT2; RACGAP1; WDR81; BCL2L11; KIF20B} \\ 
  14 & GO:0044822 & poly(A) RNA binding & \footnotesize{WDR43; DNTTIP2; ESF1; NXF1; DDX10; HEATR1; EIF3C} \\ 
  15 & GO:0044424 & intracellular part & \footnotesize{MAD2L2; PTDSS2; SMARCC1; KLHL21; TOR1B; PPRC1; SNAPC4; POLR3K; ARRDC1; SLU7; NFYB; ESF1; WDR43; ECE2; LMAN1L} \\ 
%  16 & GO:0043228 & non-membrane-bounded organelle & c & 3207 & MAD2L2; SMARCC1; KLHL21; WDR81; POLR3K; XPO1; DNTTIP2; WDR43; BCL2L11; YPEL2; HAUS6; TEKT2; BCLAF1; NF2; URB2; NR1H3; ESF1; HEATR1; CENPE; EIF4E; USP33; EBNA1BP2; PAFAH1B2; NASP; ZDHHC7; DDX24; TRAIP; RACGAP1; WDR36; POLRMT; ATXN7; KIF20B; KDM5A; POLR1E \\ 
%  17 & GO:0043232 & intracellular non-membrane-bounded organelle & c & 3207 & MAD2L2; SMARCC1; KLHL21; WDR81; POLR3K; XPO1; DNTTIP2; WDR43; BCL2L11; YPEL2; HAUS6; TEKT2; BCLAF1; NF2; URB2; NR1H3; ESF1; HEATR1; CENPE; EIF4E; USP33; EBNA1BP2; PAFAH1B2; NASP; ZDHHC7; DDX24; TRAIP; RACGAP1; WDR36; POLRMT; ATXN7; KIF20B; KDM5A; POLR1E \\ 
%  18 & GO:0051233 & spindle midzone & c &  23 & RACGAP1; CENPE; KIF20B \\ 
%  19 & GO:0003847 & 1-alkyl-2-acetylglycerophosphocholine esterase activity & m &   5 & ASPG; PAFAH1B2 \\ 
%  20 & GO:1990023 & mitotic spindle midzone & c &   5 & CENPE; KIF20B \\ 
%  21 & GO:0008017 & microtubule binding & m & 169 & WDR43; BCL2L11; RACGAP1; CENPE; WDR81; KIF20B \\ 
%  22 & GO:0071014 & post-mRNA release spliceosomal complex & c &   6 & CRNKL1; TFIP11 \\ 
%  23 & GO:1901363 & heterocyclic compound binding & m & 4675 & SMARCC1; PPRC1; POLR3K; TFIP11; NFYB; TOR1B; EIF5B; CHKA; XPO1; DNTTIP2; ZBTB10; ZBTB17; DDX10; EIF3C; BCLAF1; CRNKL1; RARS; ITPKC; RANBP2; ESF1; NXF1; HEATR1; CENPE; SNUPN; PRKCD; SPIC; WDR43; SNAPC4; EIF4E; EBNA1BP2; UBR5; THAP4; SOS1; DDX24; PFKFB3; SLU7; GEMIN5; WDR36; POLRMT; NR1H3; KIF20B; KDM5A; POLR1E \\ 
%  24 & GO:0043227 & membrane-bounded organelle & c & 9724 & MAD2L2; PTDSS2; SMARCC1; TOR1B; PPRC1; SNAPC4; SLU7; NFYB; ESF1; ECE2; LMAN1L; MIOS; NR1H3; EIF5B; POLR3K; XPO1; DNTTIP2; ZBTB10; ZBTB17; NOB1; TFIP11; HAUS6; TEKT2; BCLAF1; CRNKL1; RARS; SLC25A36; NF2; CTNNBL1; URB2; MGEA5; RANBP2; POLR1E; ITPKC; SNUPN; CTSL; NXF1; HEATR1; CENPE; CTSC; NPC1L1; ATP1B3; RTN2; PRKCD; SPIC; TNIP1; LAMP2; GFPT1; EIF4E; BCL2L11; USP33; EBNA1BP2; UBR5; PAFAH1B2; GINS3; NASP; WDR43; ZDHHC7; DDX24; RACGAP1; PFKFB3; YPEL2; TRAIP; GEMIN5; SLC9A9; WDR36; POLRMT; ATXN7; RNF216; KIF20B; KDM5A; ARRDC1 \\ 
%  25 & GO:0097159 & organic cyclic compound binding & m & 4744 & SMARCC1; PPRC1; POLR3K; TFIP11; NFYB; TOR1B; EIF5B; CHKA; XPO1; DNTTIP2; ZBTB10; ZBTB17; DDX10; EIF3C; BCLAF1; CRNKL1; RARS; ITPKC; RANBP2; ESF1; NXF1; HEATR1; CENPE; SNUPN; PRKCD; SPIC; WDR43; SNAPC4; EIF4E; EBNA1BP2; UBR5; THAP4; SOS1; DDX24; PFKFB3; SLU7; GEMIN5; WDR36; POLRMT; NR1H3; KIF20B; KDM5A; POLR1E \\ 
%  26 & GO:0005622 & intracellular & c & 11307 & MAD2L2; PTDSS2; SMARCC1; KLHL21; TOR1B; PPRC1; SNAPC4; POLR3K; ARRDC1; SLU7; NFYB; ESF1; WDR43; ECE2; LMAN1L; MIOS; NASP; NR1H3; EIF5B; CHKA; XPO1; DNTTIP2; ZBTB10; ZBTB17; NOB1; TFIP11; HAUS6; EIF3C; BCLAF1; CRNKL1; ATG3; RARS; SLC25A36; NF2; CTNNBL1; URB2; MGEA5; RANBP2; POLR1E; ITPKC; SNUPN; CTSL; NXF1; HEATR1; CENPE; CTSC; NPC1L1; ATP1B3; RTN2; PRKCD; SPIC; TNIP1; LAMP2; ATXN7; EIF4E; BCL2L11; USP33; EBNA1BP2; UBR5; PAFAH1B2; GINS3; DPH2; ZDHHC7; SOS1; DDX24; RACGAP1; PFKFB3; YPEL2; TRAIP; GEMIN5; SLC9A9; WDR36; TEKT2; POLRMT; GFPT1; RNF216; KIF20B; KDM5A; WDR81 \\ 
%  27 & GO:0043234 & protein complex & c & 3677 & MAD2L2; SMARCC1; KLHL21; NFYB; WDR81; MIOS; POLR3K; XPO1; WDR43; BCL2L11; HAUS6; EIF3C; ATG3; RARS; NF2; CTNNBL1; RANBP2; NXF1; HEATR1; CENPE; ATP1B3; SNAPC4; EIF4E; USP33; CRNKL1; NASP; GEMIN5; RACGAP1; WDR36; TEKT2; ATXN7; NR1H3; KIF20B; KDM5A; POLR1E \\ 
%  28 & GO:0000339 & RNA cap binding & m &  10 & EIF4E; SNUPN \\ 
%  29 & GO:0034062 & RNA polymerase activity & m &  40 & POLRMT; POLR3K; POLR1E \\ 
%  30 & GO:0072686 & mitotic spindle & c &  40 & RACGAP1; CENPE; KIF20B \\ 
\hline
\end{tabular}
\end{center} \label{tab:tab8}
  \end{table}


\begin{table}[htp]
\begin{center}
\caption{Deng et al (2014) Cluster 4, green annotation} \label{tab:tab5}
\begin{tabular}{|c|c|p{1.5in}|p{4in}|}
  \hline
 & go.id & name & significant \\ 
  \hline
1 & GO:0005829 & cytosol & \footnotesize{PARG; UAP1; PSMB10; TCEB1; RPLP0; EIF5; CNBP; RPS3; PSAT1; AACS; PMM1; EXOSC7; EIF3I; SET; BHMT; BHMT2} \\ 
  2 & GO:0044444 & cytoplasmic part & \footnotesize{PARG; UAP1; PSMB10; TCEB1; HSPA8; SERINC1; EIF5; CNBP; RPS3; PSAT1; GPD2; AACS; GPR137B; STIP1; PMM1; EXOSC7; VPREB3; PEX16} \\ 
  3 & GO:0055131 & C3HC4-type RING finger domain binding & \footnotesize{HSPA8; PINK1; DNAJA1} \\ 
  4 & GO:1901575 & organic substance catabolic process & \footnotesize{PSMB10; TCEB1; RPLP0; RPS3; GPD2; PINK1; EXOSC7; ALLC; BHMT; HSP90AB1; RPL13A; ATG7; CUL5; UBXN1; ZMPSTE24} \\ 
  5 & GO:0000151 & ubiquitin ligase complex & \footnotesize{DNAJA1; RNF7; UBE2C; HSPA8; FBXO15; SUGT1; DCAF4; CUL5; FBXL20} \\ 
  6 & GO:0072655 & protein localization to mitochondrion & \footnotesize{TIMM17A; BNIP3L; ARIH2; PEMT; SFN; PINK1; HSP90AA1; TIMM23} \\ 
  7 & GO:1901564 & organonitrogen compound metabolic process & \footnotesize{PSMB10; RPLP0; SERINC1; EIF5; BHMT2; PINK1; EIF3I; ALLC; BHMT; MRPL22; RPL13A; ATG7; NUDT9; VNN1; CTSA; HK1} \\ 
  8 & GO:0005737 & cytoplasm & \footnotesize{PARG; UAP1; PSMB10; TCEB1; HSPA8; SERINC1; EIF5; CNBP; RPS3; PSAT1; GPD2; AACS; GPR137B; STIP1; PMM1; EXOSC7} \\ 
  9 & GO:0044265 & cellular macromolecule catabolic process & \footnotesize{EXOSC7; SUMO2; BNIP3L; ARIH2; PSMB10; TCEB1; RPLP0; UBXN1; HSP90AB1; RPL13A; RPS3; RNF7; PINK1} \\ 
10 & GO:0023026 & MHC class II protein complex binding & \footnotesize{HSP90AB1; HSP90AA1; HSPA8} \\ 
11 & GO:0051082 & unfolded protein binding & \footnotesize{DNAJA1; PTGES3; HSPA8; HSP90AB1; HSP90AA1; NPM1} \\ 
12 & GO:0009056 & catabolic process & \footnotesize{PSMB10; TCEB1; RPLP0; RPS3; GPD2; PINK1; EXOSC7; ALLC; WDR45; HSP90AB1; RPL13A} \\ 
13 & GO:0009057 & macromolecule catabolic process & \footnotesize{EXOSC7; SUMO2; BNIP3L; ARIH2; PSMB10; TCEB1; RPLP0; AZIN1; UBXN1; HSP90AB1; RPL13A} \\ 
14 & GO:0044248 & cellular catabolic process & \footnotesize{PSMB10; TCEB1; SUMO2; RPS3; GPD2; PINK1; EXOSC7; ALLC; WDR45; HSP90AB1} \\ 
 15 & GO:0006626 & protein targeting to mitochondrion  & \footnotesize{TIMM17A; BNIP3L; ARIH2; PEMT; PINK1; HSP90AA1; TIMM23} \\ 
%  16 & GO:0030163 & protein catabolic process & b & 705 & BNIP3L; ARIH2; PSMB10; TCEB1; SUMO2; AZIN1; UBXN1; HSP90AB1; ATG7; RNF7; PINK1; CUL5; FBXL20; UBE2C; ZMPSTE24 \\ 
%  17 & GO:0044711 & single-organism biosynthetic process & b & 1499 & MRPL22; SERINC1; CNBP; BHMT2; GPD2; PINK1; PMM1; BHMT; CITED1; ISYNA1; STAG2; UAP1; BCAT1; APRT; MRPS18B; PHGDH; FDPS; DPH3; PEMT; PTGES3; AZIN1; PSAT1; GPD1L \\ 
%  18 & GO:0033477 & S-methylmethionine metabolic process & b &   2 & BHMT2; BHMT \\ 
%  19 & GO:0033528 & S-methylmethionine cycle & b &   2 & BHMT2; BHMT \\ 
%  20 & GO:0023023 & MHC protein complex binding & m &  12 & HSP90AB1; HSP90AA1; HSPA8 \\ 
%  21 & GO:0072594 & establishment of protein localization to organelle & b & 570 & HSP90AB1; TIMM17A; BNIP3L; ARIH2; PEMT; RPLP0; SFN; RPL13A; RPS3; PINK1; HSP90AA1; TIMM23; PEX16 \\ 
%  22 & GO:0044257 & cellular protein catabolic process & b & 583 & BNIP3L; ARIH2; PSMB10; TCEB1; SUMO2; UBXN1; HSP90AB1; RNF7; PINK1; CUL5; FBXL20; UBE2C; ZMPSTE24 \\ 
%  23 & GO:0031461 & cullin-RING ubiquitin ligase complex & c & 112 & RNF7; UBE2C; FBXO15; DCAF4; CUL5; FBXL20 \\ 
%  24 & GO:0043632 & modification-dependent macromolecule catabolic process & b & 515 & ARIH2; PSMB10; TCEB1; SUMO2; UBXN1; HSP90AB1; RNF7; PINK1; CUL5; FBXL20; UBE2C; ZMPSTE24 \\ 
%  25 & GO:0009331 & glycerol-3-phosphate dehydrogenase complex & c &   3 & GPD1L; GPD2 \\ 
%  26 & GO:0044267 & cellular protein metabolic process & b & 4315 & PSMB10; TCEB1; RPLP0; EIF5; SUMO2; RPS3; PINK1; PMM1; EIF3I; SET; BHMT; MRPL22; HSP90AB1; RPL13A; ATG7; CUL5; HDGF; UBXN1; ZMPSTE24; WDR45; DPH3; HSPA8; BNIP3L; ARIH2; STAG2; CTSA; UAP1; TIMM17A; MRPS18B; GPD1L; LDB1; RNF7; NPM1; PA2G4; ALPPL2; DNAJA1; PTGES3; FBXL20; UBE2C; KNG1; SFN; DCAF4; HSP90AA1; TIMM23 \\ 
%  27 & GO:0031625 & ubiquitin protein ligase binding & m & 240 & DNAJA1; UBE2C; HSPA8; UBXN1; SUMO2; PINK1; CUL5; PA2G4 \\ 
%  28 & GO:0044389 & ubiquitin-like protein ligase binding & m & 244 & DNAJA1; UBE2C; HSPA8; UBXN1; SUMO2; PINK1; CUL5; PA2G4 \\ 
%  29 & GO:0008652 & cellular amino acid biosynthetic process & b &  82 & PSAT1; BHMT2; BHMT; BCAT1; PHGDH \\ 
%  30 & GO:0006520 & cellular amino acid metabolic process & b & 393 & BHMT; PEMT; PSMB10; SERINC1; AZIN1; BCAT1; PSAT1; BHMT2; ATG7; PHGDH \\ 
%  31 & GO:0046500 & S-adenosylmethionine metabolic process & b &  18 & BHMT2; BHMT; PEMT \\ 
%  32 & GO:0051603 & proteolysis involved in cellular protein catabolic process & b & 560 & ARIH2; PSMB10; TCEB1; SUMO2; UBXN1; HSP90AB1; RNF7; PINK1; CUL5; FBXL20; UBE2C; ZMPSTE24 \\ 
%  33 & GO:0019538 & protein metabolic process & b & 4833 & PSMB10; TCEB1; RPLP0; EIF5; SUMO2; RPS3; PINK1; PMM1; EIF3I; SET; BHMT; MRPL22; HSP90AB1; RPL13A; NAALAD2; ATG7; CUL5; HDGF; UBXN1; ZMPSTE24; WDR45; DPH3; HSPA8; BNIP3L; ARIH2; STAG2; CTSA; UAP1; TIMM17A; MRPS18B; GPD1L; LDB1; RNF7; NPM1; PA2G4; ALPPL2; DNAJA1; PEMT; PTGES3; FBXL20; UBE2C; KNG1; AZIN1; SFN; DCAF4; HSP90AA1; TIMM23 \\ 
%  34 & GO:0071806 & protein transmembrane transport & b &  47 & HSP90AA1; TIMM17A; TIMM23; PEX16 \\ 
%  35 & GO:0033365 & protein localization to organelle & b & 752 & HSP90AB1; DNAJA1; BNIP3L; ARIH2; PEMT; RPLP0; SFN; RPL13A; TIMM17A; RPS3; PINK1; HSP90AA1; TIMM23; PEX16 \\ 
%  36 & GO:0044237 & cellular metabolic process & b & 8425 & PARG; UAP1; PSMB10; TCEB1; HSPA8; SERINC1; EIF5; SUMO2; CNBP; RPS3; PSAT1; GPD2; AACS; DPH3; PMM1; EXOSC7; ISYNA1; EIF3I; SET; ALLC; BHMT; BHMT2; MRPL22; HSP90AB1; RPL13A; NAALAD2; ATG7; VNN1; CUL5; CITED1; HDGF; UBXN1; ZMPSTE24; WDR45; NUDT9; RPLP0; BNIP3L; ARIH2; FDPS; STAG2; CTSA; HK1; TIMM17A; BCAT1; GPD1L; APRT; LDB1; MRPS18B; PHGDH; RNF7; HCRT; NPM1; PA2G4; ALPPL2; ACTN2; DNAJA1; PINK1; PEMT; PTMA; PTGES3; FBXL20; UBE2C; KNG1; AZIN1; SFN; DCAF4; HSP90AA1; MTA3; TIMM23 \\ 
%  37 & GO:0044419 & interspecies interaction between organisms & b & 763 & FDPS; RPLP0; BNIP3L; PSMB10; TCEB1; HSPA8; HSP90AB1; RPL13A; SET; RPS3; ATG7; CUL5; NPM1; SUGT1 \\ 
%  38 & GO:0044403 & symbiosis, encompassing mutualism through parasitism & b & 763 & FDPS; RPLP0; BNIP3L; PSMB10; TCEB1; HSPA8; HSP90AB1; RPL13A; SET; RPS3; ATG7; CUL5; NPM1; SUGT1 \\ 
%  39 & GO:0034613 & cellular protein localization & b & 1375 & WDR45; ACTN2; DNAJA1; BNIP3L; ARIH2; PEMT; CTSA; RPLP0; TIMM23; REEP1; SFN; RPL13A; TIMM17A; RPS3; HSP90AB1; PINK1; HSP90AA1; GPD1L; NPM1; PEX16 \\ 
%  40 & GO:0070727 & cellular macromolecule localization & b & 1382 & WDR45; ACTN2; DNAJA1; BNIP3L; ARIH2; PEMT; CTSA; RPLP0; TIMM23; REEP1; SFN; RPL13A; TIMM17A; RPS3; HSP90AB1; PINK1; HSP90AA1; GPD1L; NPM1; PEX16 \\ 
%  41 & GO:0044822 & poly(A) RNA binding & m & 1066 & FDPS; RPLP0; LARP4; MRPL22; RPL13A; HSPA8; EIF5; SUMO2; HSP90AB1; CNBP; RPS3; GRN; HSP90AA1; STIP1; HDGF; NPM1; PA2G4 \\ 
%  42 & GO:0044281 & small molecule metabolic process & b & 2196 & PSMB10; HSPA8; SERINC1; CNBP; BHMT2; GPD2; AACS; VNN1; PMM1; BHMT; ATG7; NUDT9; HK1; ISYNA1; CTSA; UAP1; BCAT1; APRT; PHGDH; HSP90AA1; FDPS; PEMT; PTGES3; PINK1; AZIN1; PSAT1; GPD1L \\ 
%  43 & GO:0044283 & small molecule biosynthetic process & b & 420 & FDPS; BHMT; PEMT; PTGES3; BCAT1; PSAT1; CNBP; BHMT2; PHGDH; ISYNA1 \\ 
%  44 & GO:0071704 & organic substance metabolic process & b & 8704 & PARG; UAP1; PSMB10; TCEB1; HSPA8; SERINC1; EIF5; SUMO2; CNBP; RPS3; PSAT1; GPD2; AACS; DPH3; PMM1; EXOSC7; ISYNA1; BHMT2; EIF3I; SET; ALLC; BHMT; PPT2; MRPL22; HSP90AB1; RPL13A; NAALAD2; ATG7; VNN1; CUL5; CITED1; HDGF; UBXN1; ZMPSTE24; WDR45; NUDT9; RPLP0; BNIP3L; ARIH2; FDPS; STAG2; CTSA; HK1; TIMM17A; BCAT1; GPD1L; APRT; LDB1; MRPS18B; PHGDH; RNF7; HCRT; NPM1; PA2G4; ALPPL2; ACTN2; DNAJA1; PINK1; PEMT; PTMA; PTGES3; FBXL20; UBE2C; KNG1; AZIN1; SFN; DCAF4; HSP90AA1; MTA3; TIMM23 \\ 
%  45 & GO:0016032 & viral process & b & 698 & FDPS; RPLP0; BNIP3L; PSMB10; TCEB1; HSPA8; HSP90AB1; RPL13A; SET; RPS3; ATG7; CUL5; NPM1 \\ 
%  46 & GO:0030911 & TPR domain binding & m &   5 & HSP90AB1; HSP90AA1 \\ 
%  47 & GO:1901566 & organonitrogen compound biosynthetic process & b & 1210 & EIF3I; BHMT; PEMT; RPL13A; PINK1; RPLP0; AZIN1; EIF5; BCAT1; PSAT1; APRT; RPS3; MRPS18B; PHGDH; MRPL22; BHMT2; NPM1; PA2G4 \\ 
%  48 & GO:0044764 & multi-organism cellular process & b & 708 & FDPS; RPLP0; BNIP3L; PSMB10; TCEB1; HSPA8; HSP90AB1; RPL13A; SET; RPS3; ATG7; CUL5; NPM1 \\ 
%  49 & GO:0007005 & mitochondrion organization & b & 617 & TIMM17A; BNIP3L; ARIH2; PEMT; MRPL22; MRPS18B; SFN; ATG7; PINK1; HSP90AA1; TIMM23; WDR45 \\ 
%  50 & GO:0032984 & macromolecular complex disassembly & b & 294 & ACTN2; HSPA8; MRPL22; RPLP0; MRPS18B; RPL13A; SET; RPS3 \\ 
%  51 & GO:0031466 & Cul5-RING ubiquitin ligase complex & c &   6 & CUL5; RNF7 \\ 
%  52 & GO:0030274 & LIM domain binding & m &   7 & ACTN2; LDB1 \\ 
%  53 & GO:0008172 & S-methyltransferase activity & m &   7 & BHMT2; BHMT \\ 
%  54 & GO:0009058 & biosynthetic process & b & 4979 & MRPL22; TCEB1; RPLP0; SERINC1; EIF5; SUMO2; CNBP; RPS3; PSAT1; GPD2; PINK1; PMM1; EIF3I; SET; BHMT; BHMT2; HSP90AB1; RPL13A; ATG7; CITED1; HDGF; MTA3; HSPA8; ISYNA1; FDPS; STAG2; CTSA; UAP1; BCAT1; APRT; LDB1; MRPS18B; PHGDH; GPD1L; HCRT; NPM1; PA2G4; WDR45; ACTN2; DPH3; PEMT; PTMA; PTGES3; AZIN1; SFN; HSP90AA1 \\ 
%  55 & GO:0003723 & RNA binding & m & 1405 & FDPS; RPLP0; LARP4; MRPL22; EIF3I; RPL13A; HSPA8; EIF5; SUMO2; EXOSC7; HSP90AB1; CNBP; RPS3; GRN; HSP90AA1; STIP1; HDGF; NPM1; PA2G4 \\ 
%  56 & GO:0005744 & mitochondrial inner membrane presequence translocase complex & c &   8 & TIMM17A; TIMM23 \\ 
%  57 & GO:0015450 & P-P-bond-hydrolysis-driven protein transmembrane transporter activity & m &   8 & TIMM17A; TIMM23 \\ 
%  58 & GO:0044238 & primary metabolic process & b & 8413 & PARG; UAP1; PSMB10; TCEB1; HSPA8; SERINC1; EIF5; SUMO2; CNBP; RPS3; PSAT1; GPD2; AACS; DPH3; PMM1; EXOSC7; ISYNA1; EIF3I; SET; BHMT; BHMT2; MRPL22; HSP90AB1; RPL13A; NAALAD2; ATG7; CUL5; CITED1; HDGF; UBXN1; ZMPSTE24; WDR45; NUDT9; RPLP0; BNIP3L; ARIH2; FDPS; STAG2; CTSA; HK1; TIMM17A; BCAT1; GPD1L; APRT; LDB1; MRPS18B; PHGDH; RNF7; HCRT; NPM1; PA2G4; ALPPL2; ACTN2; DNAJA1; PINK1; PEMT; PTMA; PTGES3; FBXL20; UBE2C; KNG1; AZIN1; SFN; DCAF4; HSP90AA1; MTA3; TIMM23 \\ 
%  59 & GO:0005740 & mitochondrial envelope & c & 606 & TIMM17A; BNIP3L; PEMT; MRPL22; HK1; MRPS18B; REEP1; RPS3; GPD2; PINK1; TIMM23 \\ 
%  60 & GO:0005739 & mitochondrion & c & 1450 & PARG; DRG2; DNAJA1; BNIP3L; MRPL22; PEMT; NUDT9; CTSA; HK1; BCAT1; REEP1; HSP90AB1; PINK1; TIMM17A; RPS3; MRPS18B; GPD2; GRN; TIMM23 \\ 
%  61 & GO:0032592 & integral component of mitochondrial membrane & c &  36 & PINK1; TIMM17A; TIMM23 \\ 
%  62 & GO:0098573 & intrinsic component of mitochondrial membrane & c &  37 & PINK1; TIMM17A; TIMM23 \\ 
%  63 & GO:0019005 & SCF ubiquitin ligase complex & c &  38 & RNF7; FBXL20; FBXO15 \\ 
   \hline
\end{tabular} \end{center} \label{tab:tab5}
  \end{table}
\clearpage






\begin{table}[htp]
\begin{center}
\caption{Deng et al (2014) Cluster 5, purple annotation} \label{tab:tab6}
\begin{tabular}{|c|c|p{1.5in}|p{4in}|}
  \hline
 & go.id & name & significant \\ 
  \hline
1 & GO:0044710 & single-organism metabolic process & \footnotesize{PCK2; SAT1; EPHX2; NFATC4; CKB; PRDX6; MSH2; EPHA4; PROS1; PDGFRA; PRDX1; UBE2L6; POGLUT1; FABP5; AKAP12; TDGF1; FBP2; SOX2} \\ 
2 & GO:0006950 & response to stress & \footnotesize{EPHX2; NFATC4; PRDX6; MSH2; EPHA4; PROS1; PDGFRA; PRDX1; UBE2L6; FABP5; TDGF1; SOX2} \\ 
3 & GO:0065010 & extracellular membrane-bounded organelle & \footnotesize{PCK2; EPHX2; MFGE8; CKB; PRDX6; PROS1; PRDX1; POGLUT1; FABP5; FBP2; TRAP1; PLOD2; DHRS4} \\ 
4 & GO:0070062 & extracellular exosome  & \footnotesize{PCK2; EPHX2; MFGE8; CKB; PRDX6; PROS1; PRDX1; POGLUT1; FABP5; FBP2; TRAP1; PLOD2; DHRS4; MARCKS; DPP4; PRKCI; RAC2; IDH1} \\ 
5 & GO:0043230 & extracellular organelle & \footnotesize{PCK2; EPHX2; MFGE8; CKB; PRDX6; PROS1; PRDX1; POGLUT1; FABP5; FBP2; TRAP1; PLOD2; DHRS4; MARCKS; DPP4} \\ 
6 & GO:1903561 & extracellular vesicle & \footnotesize{PCK2; EPHX2; MFGE8; CKB; PRDX6; PROS1; PRDX1; POGLUT1; FABP5; FBP2; TRAP1; PLOD2; DHRS4; MARCKS; DPP4; PRKCI} \\ 
7 & GO:0042221 & response to chemical & \footnotesize{EPHX2; NFATC4; MFGE8; PRDX6; EPHA4; PROS1; PDGFRA; PRDX1; UBE2L6; TDGF1; SOX2} \\ 
8 & GO:0031988 & membrane-bounded vesicle & \footnotesize{PCK2; EPHX2; MFGE8; CKB; PRDX6; PROS1; PRDX1; POGLUT1; FABP5; FBP2; TRAP1; PLOD2; DHRS4; SPARC} \\ 
9 & GO:0031982 & vesicle & \footnotesize{PCK2; EPHX2; MFGE8; CKB; PRDX6; PROS1; PRDX1; POGLUT1; FABP5; FBP2; TRAP1; PLOD2; DHRS4; SPARC} \\ 
10 & GO:0001525 & angiogenesis & \footnotesize{SAT1; PDGFRA; BMP4; NFATC4; MFGE8; FN1; MEIS1; SPARC; COL4A2; COL4A1; FGF10; TDGF1} \\ 
11 & GO:0048514 & blood vessel morphogenesis & \footnotesize{SAT1; PDGFRA; BMP4; NFATC4; MFGE8; FN1; ZFP36L1; MEIS1; SPARC; COL4A2; COL4A1; FGF10; TDGF1} \\ 
12 & GO:0001944 & vasculature development & \footnotesize{SAT1; PDGFRA; BMP4; NFATC4; MFGE8; FN1; ZFP36L1; MEIS1; PDPN; SPARC; COL4A2; COL4A1; FGF10; TDGF1} \\ 
13 & GO:0006979 & response to oxidative stress & \footnotesize{TAT; PDGFRA; BMP4; ETV5; TRAP1; PRDX6; IDH1; PARP1; AQP8; PRDX1; CRYGD} \\ 
14 & GO:0009725 & response to hormone & \footnotesize{PRKCI; GJA1; PDGFRA; BMP4; MFGE8; TAT; PLOD2; SPP1; IDH1} \\ 
15 & GO:0030198 & extracellular matrix organization & \footnotesize{PDGFRA; BMP4; JAM2; FN1; PLOD2; SPARC; SPP1; COL4A2; COL4A1; SERPINH1; DPP4} \\ 
%  16 & GO:0043062 & extracellular structure organization & b & 368 & PDGFRA; BMP4; JAM2; FN1; PLOD2; SPARC; SPP1; COL4A2; COL4A1; SERPINH1; DPP4 \\ 
%  17 & GO:0001568 & blood vessel development & b & 532 & SAT1; PDGFRA; BMP4; NFATC4; MFGE8; FN1; ZFP36L1; MEIS1; SPARC; COL4A2; COL4A1; FGF10; TDGF1 \\ 
%  18 & GO:0001654 & eye development & b & 331 & PRKCI; PDGFRA; BMP4; GDF3; MEIS1; CDK4; COL4A1; SOX2; FGF10; CRYGD \\ 
%  19 & GO:0014070 & response to organic cyclic compound & b & 749 & TAT; PDGFRA; BMP4; MFGE8; IDH1; HNF4A; PARP1; AQP8; ASNS; SPARC; CDK4; ZC3HAV1; SPP1; FGF10; RALB \\ 
%  20 & GO:0009719 & response to endogenous stimulus & b & 1455 & PDGFRA; MFGE8; TDGF1; TAT; PLOD2; SPARC; CDK4; FGF10; PRKCI; EIF4EBP1; IDH1; PARP1; AQP8; COL4A2; COL4A1; GJA1; BMP4; LMO2; HNF4A; ASNS; GDF3; SPP1 \\ 
%  21 & GO:0072358 & cardiovascular system development & b & 857 & GJA1; PDGFRA; BMP4; NFATC4; MFGE8; FN1; SAT1; ZFP36L1; MEIS1; PDPN; POGLUT1; SPARC; COL4A2; COL4A1; FGF10; TDGF1 \\ 
%  22 & GO:0072359 & circulatory system development & b & 857 & GJA1; PDGFRA; BMP4; NFATC4; MFGE8; FN1; SAT1; ZFP36L1; MEIS1; PDPN; POGLUT1; SPARC; COL4A2; COL4A1; FGF10; TDGF1 \\ 
%  23 & GO:0043231 & intracellular membrane-bounded organelle & c & 8750 & PCK2; EPHX2; RPAP1; NFATC4; NFU1; CKB; PRDX6; MSH2; EPHA4; PROS1; PDGFRA; PRDX1; UBE2L6; POGLUT1; FABP5; TDGF1; FBP2; SOX2; TXNDC12; CRYGD; TAT; ORC3; TRAP1; PLOD2; POLDIP2; DHRS4; MEIS1; RNF130; SPARC; MARCKS; TBX15; PRSS35; EIF4EBP1; FGF10; DPP4; ZC3HAV1; PRKCI; RAC2; TET1; IDH1; SLC24A5; ZFP36L1; AQP8; PARP1; MCM5; ATG13; COL4A2; COL4A1; GLRX; ACO1; GCAT; MTCH2; GJA1; AGTRAP; ETV5; SH3BGRL3; FN1; FBXO3; CCT6A; LMO2; HNF4A; GLUD1; AK4; IGF2BP1; RND3; CDK4; PSMB9; NUCKS1; MESDC2; SERPINH1; AHCY; RALB \\ 
%  24 & GO:0043227 & membrane-bounded organelle & c & 9724 & PCK2; EPHX2; RPAP1; NFATC4; MFGE8; CKB; PRDX6; MSH2; EPHA4; PROS1; PDGFRA; PRDX1; UBE2L6; POGLUT1; FABP5; TDGF1; FBP2; SOX2; TXNDC12; CRYGD; TAT; ORC3; POLDIP2; TRAP1; PLOD2; SPP1; DHRS4; AGTRAP; MEIS1; RNF130; SPARC; MARCKS; TBX15; PRSS35; EIF4EBP1; FGF10; CCT6A; PGM1; ZC3HAV1; PRKCI; RAC2; TET1; IDH1; SLC24A5; PARP1; AQP8; ZFP36L1; MCM5; ATG13; COL4A2; COL4A1; GLRX; ACO1; GCAT; MTCH2; SMPDL3B; GJA1; BMP4; ETV5; SH3BGRL3; FN1; FBXO3; DPP4; LMO2; HNF4A; GLUD1; AK4; IGF2BP1; RND3; CDK4; PSMB9; NUCKS1; MESDC2; SERPINH1; AHCY; RALB; NFU1 \\ 
%  25 & GO:0044444 & cytoplasmic part & c & 6844 & PCK2; SAT1; EPHX2; NFATC4; NFU1; CKB; PRDX6; EPHA4; PROS1; PDGFRA; PRDX1; UBE2L6; POGLUT1; FABP5; AKAP12; TDGF1; FBP2; SOX2; TXNDC12; TAT; TRAP1; PLOD2; SPP1; DHRS4; IGF2BP1; SPARC; MARCKS; ZC3HAV1; PRSS35; EIF4EBP1; AGTRAP; DPP4; PGM1; PRKCI; RAC2; IDH1; SLC24A5; ZFP36L1; PARP1; ATG13; COL4A2; COL4A1; GLRX; ACO1; GCAT; GJA1; UPP1; MTCH2; FN1; CCT6A; ASNS; GLUD1; AK4; RND3; CDK4; PSMB9; POLDIP2; MESDC2; SERPINH1; AHCY; RALB \\ 
%  26 & GO:0007369 & gastrulation & b & 171 & BMP4; FN1; GDF3; HNF4A; COL4A2; SOX2; TDGF1 \\ 
%  27 & GO:0070887 & cellular response to chemical stimulus & b & 2305 & EPHX2; NFATC4; PDGFRA; PRDX1; UBE2L6; TDGF1; CRYGD; TRAP1; PLOD2; SPARC; ZC3HAV1; SPP1; FGF10; PRKCI; IFITM2; RAC2; PARP1; AQP8; COL4A2; COL4A1; BMP4; ETV5; LMO2; HNF4A; ASNS; GDF3; EIF4EBP1; AHCY; RALB \\ 
%  28 & GO:0048144 & fibroblast proliferation & b &  78 & PDGFRA; FGF10; CDK4; TDGF1; FN1 \\ 
%  29 & GO:0048513 & organ development & b & 2833 & PDGFRA; NFATC4; CKB; MSH2; EPHA4; TDGF1; PDPN; SOX2; CRYGD; GDF3; SPP1; GJB5; MEIS1; SPARC; TBX15; EIF4EBP1; FGF10; PRKCI; IDH1; PARP1; ZFP36L1; COL4A1; IGF2BP1; GJA1; BMP4; ETV5; LMO2; HNF4A; ASNS; GLUD1; AK4; CDK4; SERPINH1 \\ 
%  30 & GO:0044712 & single-organism catabolic process & b & 1097 & TAT; SMPDL3B; EPHX2; PRDX1; PLOD2; LIPH; IDH1; PGM1; GLUD1; UPP1; ATG13; FABP5; COL4A2; COL4A1; PRDX6; GCAT; AHCY; RALB \\ 
%  31 & GO:0048545 & response to steroid hormone & b & 381 & TAT; PDGFRA; BMP4; MFGE8; IDH1; HNF4A; SPARC; CDK4; SPP1; FGF10 \\ 
%  32 & GO:0000302 & response to reactive oxygen species & b & 182 & PDGFRA; BMP4; PRDX6; PARP1; AQP8; PRDX1; CRYGD \\ 
%  33 & GO:0048037 & cofactor binding & m & 249 & TAT; IDH1; PARP1; GLUD1; ASNS; GCAT; AHCY; LMO2 \\ 
%  34 & GO:0034599 & cellular response to oxidative stress & b & 188 & PDGFRA; BMP4; ETV5; TRAP1; PARP1; PRDX1; CRYGD \\ 
%  35 & GO:0016763 & transferase activity, transferring pentosyl groups & m &  45 & POGLUT1; UPP1; PARP1; ZC3HAV1 \\ 
%  36 & GO:0044763 & single-organism cellular process & b & 10289 & EPHX2; NFATC4; MFGE8; CKB; PRDX6; MSH2; EPHA4; PROS1; PDGFRA; PRDX1; UBE2L6; POGLUT1; FABP5; AKAP12; TDGF1; MESDC2; PDPN; SOX2; TXNDC12; CRYGD; TAT; ORC3; TRAP1; PLOD2; SPP1; DHRS4; AGTRAP; GJB5; GJB4; MEIS1; RNF130; SPARC; MARCKS; ZC3HAV1; SLC13A5; EIF4EBP1; FGF10; CCT6A; PGM1; PRKCI; IFITM2; RAC2; JAM2; PYY; TET1; IGF2BP1; IDH1; SLC24A5; ZFP36L1; AQP8; PARP1; MCM5; ATG13; COL4A2; COL4A1; GLRX; ACO1; GCAT; MTCH2; SMPDL3B; GJA1; UPP1; BMP4; ETV5; SH3BGRL3; FN1; DPP4; HNF4A; ASNS; GLUD1; AK4; GDF3; RND3; CDK4; PSMB9; POLDIP2; SERPINH1; AHCY; RALB \\ 
%  37 & GO:0005739 & mitochondrion & c & 1450 & PCK2; GJA1; TAT; MTCH2; NFU1; CKB; POLDIP2; IDH1; EPHA4; TRAP1; PARP1; GLUD1; AK4; SPARC; ATG13; PRDX1; PRSS35; GLRX; ACO1; GCAT; DHRS4 \\ 
%  38 & GO:0048598 & embryonic morphogenesis & b & 567 & PDGFRA; BMP4; FN1; GDF3; HNF4A; GJB5; ZFP36L1; COL4A2; SOX2; TDGF1; FGF10; TBX15 \\ 
%  39 & GO:0005922 & connexon complex & c &  19 & GJA1; GJB5; GJB4 \\ 
%  40 & GO:0044707 & single-multicellular organism process & b & 5631 & SAT1; EPHX2; NFATC4; MFGE8; CKB; MSH2; EPHA4; PROS1; PDGFRA; PRDX1; UBE2L6; POGLUT1; TDGF1; PDPN; SOX2; CRYGD; GDF3; SPP1; AGTRAP; GJB5; GJB4; MEIS1; IGF2BP1; SPARC; TBX15; EIF4EBP1; FGF10; DPP4; ZC3HAV1; PRKCI; RAC2; JAM2; PYY; TET1; IDH1; PARP1; ZFP36L1; COL4A2; COL4A1; ACO1; GJA1; BMP4; ETV5; SH3BGRL3; FN1; LMO2; HNF4A; ASNS; GLUD1; AK4; CDK4; SERPINH1 \\ 
%  41 & GO:0048731 & system development & b & 3886 & SAT1; PDGFRA; NFATC4; MFGE8; CKB; MSH2; EPHA4; PRDX1; POGLUT1; TDGF1; PDPN; SOX2; CRYGD; GDF3; SPP1; GJB5; MEIS1; SPARC; TBX15; EIF4EBP1; FGF10; PRKCI; RAC2; IDH1; PARP1; ZFP36L1; COL4A2; COL4A1; IGF2BP1; GJA1; BMP4; ETV5; FN1; LMO2; HNF4A; ASNS; GLUD1; AK4; CDK4; SERPINH1 \\ 
%  42 & GO:0007423 & sensory organ development & b & 499 & PRKCI; PDGFRA; BMP4; GDF3; MEIS1; SPARC; CDK4; COL4A1; SOX2; FGF10; CRYGD \\ 
%  43 & GO:0051287 & NAD binding & m &  53 & IDH1; AHCY; PARP1; GLUD1 \\ 
%  44 & GO:0010033 & response to organic substance & b & 2520 & PDGFRA; MFGE8; PROS1; UBE2L6; TDGF1; SOX2; TAT; PLOD2; SPP1; SPARC; ZC3HAV1; EIF4EBP1; FGF10; PRKCI; IFITM2; IDH1; PARP1; AQP8; COL4A2; COL4A1; GJA1; BMP4; LMO2; HNF4A; ASNS; GDF3; CDK4; SERPINH1; RALB \\ 
%  45 & GO:0005587 & collagen type IV trimer & c &   6 & COL4A2; COL4A1 \\ 
%  46 & GO:0009653 & anatomical structure morphogenesis & b & 2422 & SAT1; PDGFRA; NFATC4; MFGE8; EPHA4; TDGF1; SOX2; GDF3; POLDIP2; GJB5; MEIS1; SPARC; TBX15; FGF10; PRKCI; RAC2; TET1; ZFP36L1; PDPN; COL4A2; COL4A1; GJA1; BMP4; ETV5; FN1; HNF4A; SPP1; SERPINH1 \\ 
%  47 & GO:0048856 & anatomical structure development & b & 4609 & SAT1; PDGFRA; NFATC4; MFGE8; CKB; MSH2; EPHA4; MESDC2; PRDX1; POGLUT1; FABP5; TDGF1; PDPN; SOX2; CRYGD; GDF3; SPP1; GJB5; MEIS1; SPARC; TBX15; EIF4EBP1; FGF10; PRKCI; RAC2; TET1; IDH1; PARP1; ZFP36L1; COL4A2; COL4A1; IGF2BP1; GJA1; BMP4; ETV5; FN1; LMO2; HNF4A; ASNS; GLUD1; AK4; CDK4; POLDIP2; SERPINH1 \\ 
%  48 & GO:0043229 & intracellular organelle & c & 9562 & PCK2; EPHX2; RPAP1; NFATC4; NFU1; CKB; PRDX6; MSH2; EPHA4; PROS1; PDGFRA; PRDX1; UBE2L6; POGLUT1; FABP5; AKAP12; TDGF1; FBP2; SOX2; TXNDC12; CRYGD; TAT; ORC3; TRAP1; PLOD2; POLDIP2; DHRS4; MEIS1; RNF130; SPARC; MARCKS; TBX15; PRSS35; EIF4EBP1; FGF10; DPP4; PGM1; ZC3HAV1; PRKCI; RAC2; TET1; IDH1; SLC24A5; ZFP36L1; AQP8; PARP1; MCM5; ATG13; COL4A2; COL4A1; GLRX; ACO1; GCAT; MTCH2; GJA1; AGTRAP; ETV5; SH3BGRL3; FN1; FBXO3; CCT6A; LMO2; HNF4A; GLUD1; AK4; IGF2BP1; RND3; CDK4; PSMB9; NUCKS1; MESDC2; SERPINH1; AHCY; RALB \\ 
%  49 & GO:0005921 & gap junction & c &  28 & GJA1; GJB5; GJB4 \\ 
%  50 & GO:0008283 & cell proliferation & b & 1729 & SAT1; PDGFRA; MFGE8; PRDX1; TDGF1; SOX2; PDPN; SPARC; FGF10; DPP4; PRKCI; RAC2; PYY; ORC3; ZFP36L1; IGF2BP1; GJA1; BMP4; ETV5; FN1; HNF4A; CDK4 \\ 
%  51 & GO:0098642 & network-forming collagen trimer & c &   7 & COL4A2; COL4A1 \\ 
%  52 & GO:0098645 & collagen network & c &   7 & COL4A2; COL4A1 \\ 
%  53 & GO:0009058 & biosynthetic process & b & 4979 & PCK2; SAT1; PDGFRA; NFATC4; NFU1; ORC3; PRDX1; UBE2L6; POGLUT1; FABP5; AKAP12; TDGF1; FBP2; SOX2; RPAP1; TAT; TRAP1; GDF3; MEIS1; IGF2BP1; TBX15; EIF4EBP1; FGF10; PGM1; PRKCI; TET1; IDH1; SLC24A5; PARP1; ZFP36L1; MCM5; COL4A2; ACO1; GCAT; GJA1; UPP1; BMP4; ETV5; LMO2; HNF4A; ASNS; GLUD1; AK4; CDK4; SERPINH1; AHCY \\ 
%  54 & GO:0048646 & anatomical structure formation involved in morphogenesis & b & 1075 & SAT1; PDGFRA; BMP4; NFATC4; MFGE8; FN1; GDF3; MEIS1; PDPN; TET1; SPARC; COL4A2; COL4A1; SOX2; FGF10; TDGF1 \\ 
%  55 & GO:0009056 & catabolic process & b & 2001 & SAT1; EPHX2; PRDX6; PRDX1; UBE2L6; FABP5; TAT; PLOD2; ZC3HAV1; LIPH; IDH1; ZFP36L1; PGM1; ATG13; COL4A2; COL4A1; GCAT; SMPDL3B; GJA1; UPP1; GLUD1; PSMB9; AHCY; RALB \\ 
%  56 & GO:0098651 & basement membrane collagen trimer & c &   8 & COL4A2; COL4A1 \\ 
%  57 & GO:0044237 & cellular metabolic process & b & 8425 & PCK2; SAT1; EPHX2; NFATC4; CKB; PRDX6; MSH2; EPHA4; PROS1; PDGFRA; PRDX1; UBE2L6; POGLUT1; FABP5; AKAP12; TDGF1; FBP2; PDPN; SOX2; RPAP1; TAT; ORC3; TRAP1; PLOD2; DHRS4; MEIS1; RNF130; MARCKS; TBX15; EIF4EBP1; FGF10; CCT6A; PGM1; ZC3HAV1; PRKCI; RAC2; TET1; IGF2BP1; IDH1; SLC24A5; PARP1; ZFP36L1; MCM5; ATG13; COL4A2; GLRX; ACO1; GCAT; SMPDL3B; GJA1; UPP1; BMP4; ETV5; FN1; LMO2; HNF4A; ASNS; GLUD1; AK4; GDF3; FBXO3; CDK4; PSMB9; MESDC2; SERPINH1; AHCY; RALB \\ 
%  58 & GO:0005615 & extracellular space & c & 1105 & SMPDL3B; PYY; BMP4; FN1; MFGE8; CKB; GDF3; LIPH; PROS1; PRDX1; SPARC; SPP1; TDGF1; PRDX6; SERPINH1; FGF10 \\ 
%  59 & GO:0065008 & regulation of biological quality & b & 3099 & EPHX2; NFATC4; CKB; PROS1; PDGFRA; PRDX1; IGF2BP1; AGTRAP; PDPN; SPARC; MARCKS; ZC3HAV1; SPP1; FGF10; RAC2; JAM2; PYY; PARP1; AQP8; ZFP36L1; GLRX; ACO1; MTCH2; GJA1; BMP4; ETV5; SH3BGRL3; FN1; HNF4A; GLUD1; TXNDC12; CDK4 \\ 
%  60 & GO:0071704 & organic substance metabolic process & b & 8704 & PCK2; SAT1; EPHX2; NFATC4; CKB; PRDX6; MSH2; EPHA4; PROS1; PDGFRA; PRDX1; UBE2L6; POGLUT1; FABP5; AKAP12; TDGF1; FBP2; PDPN; SOX2; RPAP1; TAT; ORC3; TRAP1; PLOD2; DHRS4; MEIS1; RNF130; TBX15; EIF4EBP1; FGF10; DPP4; PGM1; ZC3HAV1; PRKCI; RAC2; TET1; LIPH; IDH1; SLC24A5; PARP1; ZFP36L1; MCM5; COL4A2; COL4A1; GLRX; ACO1; GCAT; IGF2BP1; SMPDL3B; GJA1; UPP1; BMP4; ETV5; FN1; CCT6A; LMO2; HNF4A; ASNS; GLUD1; AK4; GDF3; FBXO3; CDK4; PSMB9; MESDC2; SERPINH1; AHCY; RALB \\ 
   \hline
\end{tabular} \label{tab:tab6}
  \end{center} 
  \end{table}

\clearpage


\begin{table}[htp]
\begin{center}
\caption{Deng et al (2014) Cluster 6, orange annotation} \label{tab:tab7}
\begin{tabular}{|c|c|p{1.5in}|p{4in}|}
  \hline
& go.id & name & genes \\ 
  \hline
1 & GO:0065010 & extracellular membrane-bounded organelle  & \footnotesize{MYH10; SLC2A3; GM2A; TSPAN8; ACTG1; SDC4; TINAGL1; CRYAB; MSN; FABP3; PDZK1IP1; PRSS8; S100A11; DAB2; KRT8; LCP1; UGP2}   \\ 
2 & GO:0070062 & extracellular exosome &  \footnotesize{MYH10; SLC2A3; GM2A; TSPAN8; ACTG1; SDC4; TINAGL1; CRYAB; MSN; FABP3; PDZK1IP1; PRSS8; S100A11; DAB2; KRT8; LCP1; UGP2}\\ 
3 & GO:0043230 & extracellular organelle & \footnotesize{MYH10; SLC2A3; GM2A; TSPAN8; ACTG1; SDC4; TINAGL1; CRYAB; MSN; FABP3; PDZK1IP1; PRSS8; S100A11} \\ 
4 & GO:1903561 & extracellular vesicle & \footnotesize{MYH10; SLC2A3; GM2A; TSPAN8; ACTG1; SDC4; TINAGL1; CRYAB; MSN; FABP3; PDZK1IP1; PRSS8; S100A11; DAB2; KRT8}  \\ 
5 & GO:0031988 & membrane-bounded vesicle & \footnotesize{MYH10; SLC2A3; GM2A; TSPAN8; ACTG1; TMSB4X; SDC4; TINAGL1; CRYAB; MSN; FABP3; PDZK1IP1; PRSS8; S100A11; DAB2}  \\ 
6 & GO:0031982 & vesicle & \footnotesize{MYH10; SLC2A3; GM2A; TSPAN8; ACTG1; TMSB4X; SDC4; TINAGL1; CRYAB; MSN; FABP3; PDZK1IP1; PRSS8; S100A11; DAB2; KRT8} \\ 
7 & GO:0008092 & cytoskeletal protein binding & \footnotesize{MYH10; TPM4; TMSB4X; CRYAB; MSN; TMSB10; FABP3; NDRG1; CALM1; FMNL2; MYH9; CAP1; TPM1; CDH1} \\ 
8 & GO:0015629 & actin cytoskeleton & \footnotesize{MYH10; CLIC4; MYH9; MYL12B; WDR1; CNN2; ARPC2; AHNAK; ACTN4; CRYAB; CAP1; TPM1; DSTN; ARPC5; TPM4} \\ 
9 & GO:0003779 & actin binding & \footnotesize{MYH10; TPM4; WDR1; CNN2; FMNL2; ARPC2; MYH9; CAP1; TPM1}  \\ 
10 & GO:0048468 & cell development & \footnotesize{MYH10; CAPG; ACTG1; WDR1; CNN2; FMNL2; MYH9; ACTN4; SDC4; CAP1; TPM1; DSTN} \\ 
11 & GO:0030036 & actin cytoskeleton organization & \footnotesize{MYH10; CAPG; ACTG1; WDR1; CNN2; FMNL2; MYH9; ACTN4; SDC4; CAP1; TPM1} \\ 
12 & GO:0032432 & actin filament bundle & \footnotesize{MYH10; TPM4; MYL12B; CNN2; MYH9; CRYAB; TPM1; ACTN4; LCP1} \\ 
13 & GO:0005912 & adherens junction & \footnotesize{TJP2; MYH9; ACTG1; CNN2; ARPC2; AHNAK; ACTN4; SDC4} \\ 
14 & GO:0070161 & anchoring junction & \footnotesize{TJP2; MYH9; ACTG1; CNN2; ARPC2; AHNAK; ACTN4; SDC4} \\ 
15 & GO:0005925 & focal adhesion & \footnotesize{MYH9; ACTG1; CNN2; ARPC2; AHNAK; ACTN4; SDC4; CAP1; ARPC5} \\ 
\hline
\end{tabular}
\end{center} \label{tab:tab7}
  \end{table}
\clearpage

%For the reads expression data, we got the following table
%
%\begin{table}
%\newpage
%\begin{center}
%\begin{tabular}{|c|c|p{1.9in}|p{2.5in}|} 
%\hline
%Cluster & Gene names & Proteins  & Summary \\
%\hline
%
% \multirow{3}{4em}{cluster 1,red (adipose, breast)}  &  ENSG00000170323 & fatty acid binding protein 4, adipocyte & encodes the fatty acid binding protein found in adipocytes, involved in fatty acid uptake, transport, and metabolism.  \\ 
% 					      & ENSG00000181092 & adiponectin, C1Q and collagen domain containing & expressed in adipose tissue exclusively, mutations in this gene are associated with adiponectin deficiency\\
%					      & ENSG00000189058 & apolipoprotein D & encodes a component of high density lipoprotein that has no marked similarity to other apolipoprotein sequences, associated with lipoprotein metabolism. \\
% \hline
% \multirow{3}{4em}{cluster 2, blue (testis, ovary, prostrate)} & ENSG00000122304 & protamine 2 & substitute for histones in the chromatin of sperm during the haploid phase of spermatogenesis, and are the major DNA-binding proteins in the nucleus of sperm in many vertebrates. \\
% 					    & ENSG00000167751 & kallikrein-related peptidase 2 & primarily expressed in prostatic tissue and is responsible for cleaving pro-prostate-specific antigen into its enzymatically active form; may be a prognostic maker for prostate cancer risk \\
%					    &  ENSG00000010318 & PHD finger protein 7 & This gene is expressed in the testis in Sertoli cells but not germ cells, has been implicated in the transcriptional regulation of spermatogenesis\\
% \hline
% \multirow{3}{4em}{cluster 3, shallow blue (colon and esophagus)} & ENSG00000198804 & cytochrome c oxidase subunit I & NA \\
% 					    & ENSG00000198888 & NADH dehydrogenase, subunit 1 (complex I) & NA.\\
%					    & ENSG00000198727  & cytochrome b & NA. \\
%\hline
%\end{tabular}
% \end{center}
% \end{table}
% 
% 
%\begin{table}
%\newpage
%\begin{center}
%\begin{tabular}{|c|c|p{1.9in}|p{2.5in}|} 
%\hline
%Cluster & Gene names & Proteins  & Summary \\
%\hline
%
% \multirow{3}{4em}{cluster 4, black (brain)} & ENSG00000163017  & actin, gamma 2, smooth muscle, enteric & actins are highly conserved proteins that are involved in various types of cell motility and in the maintenance of the cytoskeleton: beta and gamma actins co-exist in most cell types as components of the cytoskeleton and as mediators of internal cell motility. \\
% 					     & ENSG00000133392 & myosin, heavy chain 11, smooth muscle & functions as a major contractile protein, converting chemical energy into mechanical energy through the hydrolysis of ATP\\
%					     & ENSG00000065534 & myosin light chain kinase & encodes myosin light chain kinase which is a calcium/calmodulin dependent enzyme, functions to stabilize unphosphorylated myosin filaments \\	
%\hline				     
% \multirow{3}{4em}{cluster 5,light blue (artery)} & ENSG00000197971 & myelin basic protein & major constituent of the myelin sheath of oligodendrocytes and Schwann cells in the nervous system \\.
% 					     & ENSG00000131095 & glial fibrillary acidic protein & encodes one of the major intermediate filament proteins of mature astrocytes, mutations casuses Alexander disease. \\
%					     & ENSG00000172179 & prolactin & encodes the anterior pituitary hormone prolactin, secreted hormone is a growth regulator for many tissues, including cells of the immune system, may also play a role in cell survival by suppressing apoptosis, and essential for lactation\\					     
% \hline
% \multirow{3}{4em}{cluster 6,deep blue (muscle heart)} & ENSG00000143632 & actin, alpha 1, skeletal muscle & produces highly conserved proteins that play a role in cell motility, structure and integrity, mutations cause nemaline myopathy type 3, congenital myopathy, diseases  leading to muscle fibre defects \\
% 					    &  ENSG00000104879 & creatine kinase, muscle & protein encoded is cytoplasmic enzyme involved in energy homeostasis and serum marker for myocardial infarction. \\
%					    & ENSG00000092054 & myosin, heavy chain 7, cardiac muscle, beta & This gene encodes the beta (or slow) heavy chain subunit of cardiac myosin. It is expressed predominantly in normal human ventricle, also expressed in skeletal muscle tissues rich in slow-twitch type I muscle fibers. Mutations in this gene are associated with familial hypertrophic cardiomyopathy, myosin storage myopathy, dilated cardiomyopathy, and Laing early-onset distal myopathy. \\
% \hline
% \multirow{3}{4em}{cluster 7,dark brown (brain)} & ENSG00000168878 & surfactant protein B & encodes the pulmonary-associated surfactant protein B (SPB), an amphipathic surfactant protein essential for lung function and homeostasis after birth, creted by the alveolar cells of the lung and maintains the stability of pulmonary tissue by reducing the surface tension of fluids that coat the lung, mutations associated with fatal respiratory distress in the neonatal period.\\
% 					    & ENSG00000185303 & surfactant protein A2 & encoding pulmonary-surfactant associated proteins (SFTPA) located on chromosome 10. Mutations in this gene and a highly similar gene located nearby, which affect the highly conserved carbohydrate recognition domain, are associated with idiopathic pulmonary fibrosis.\\
%					    & ENSG00000168484 & surfactant protein C & encodes the pulmonary-associated surfactant protein C (SPC), an extremely hydrophobic surfactant protein essential for lung function and homeostasis after birth, associated with interstitial lung disease in older infants, children, and adults\\
%\hline	
% \end{tabular}
% \end{center}
%\end{table}
%
%
%\begin{table}
%\newpage
%\begin{center}
%\begin{tabular}{|c|c|p{1.9in}|p{2.5in}|} 
%\hline
%Cluster & Gene names & Proteins  & Summary \\
%\hline
%
%		    
% \multirow{3}{4em}{cluster 8, shallow yellow (skin stomach)} & ENSG00000186395 & keratin 10, type I & encodes a member of the type I (acidic) cytokeratin family, mutations associated with epidermolytic hyperkeratosis. \\
% 					    & ENSG00000167768 & keratin 1, type II  & specifically expressed in the spinous and granular layers of the epidermis with family member KRT10 and mutations in these genes have been associated with bullous congenital ichthyosiform erythroderma. \\
%					    & ENSG00000171195 & mucin 7, secreted & encodes a small salivary mucin, which is thought to play a role in facilitating the clearance of bacteria in the oral cavity and to aid in mastication, speech, and swallowing, associated with susceptibility to asthma.\\
%\hline				   
% \multirow{3}{4em}{cluster 9, yellow (cell EBV)} & ENSG00000096088 & progastricsin (pepsinogen C)& encodes an aspartic proteinase that belongs to the peptidase family A1. The encoded protein is a digestive enzyme that is produced in the stomach and constitutes a major component of the gastric mucosa  \\
% 					    & ENSG00000182333 & lipase, gastric & encodes gastric lipase, an enzyme involved in the digestion of dietary triglycerides in the gastrointestinal tract, and responsible for 30 $\%$ of fat digestion processes occurring in human\\
%					    & ENSG00000229859 & pepsinogen 3, group I (pepsinogen A) & encodes a protein precursor of the digestive enzyme pepsin, a member of the peptidase A1 family of endopeptidases, biomarker for atrophic gastritis and gastric cancer \\
%\hline					    
%\multirow{3}{4em}{cluster 11,cyan cluster (cells fibroblasts)} & ENSG00000115414 & fibronectin 1 & Fibronectin is involved in cell adhesion, embryogenesis, blood coagulation, host defense, and metastasis \\
% 					      & ENSG00000108821 & collagen, type I, alpha 1 & Mutations in this gene associated with osteogenesis imperfecta types I-IV, Ehlers-Danlos syndrome type and Classical type, Caffey Disease \\
%					      & ENSG00000164692 & collagen, type I, alpha 2 & Same as above \\
% \hline
% \multirow{3}{4em}{cluster 11,cyan cluster (cells fibroblasts)} & ENSG00000115414 & fibronectin 1 & Fibronectin is involved in cell adhesion, embryogenesis, blood coagulation, host defense, and metastasis \\
% 					      & ENSG00000108821 & collagen, type I, alpha 1 & Mutations in this gene associated with osteogenesis imperfecta types I-IV, Ehlers-Danlos syndrome type and Classical type, Caffey Disease \\
%					      & ENSG00000164692 & collagen, type I, alpha 2 & Same as above \\
%\hline		
% \end{tabular}
% \end{center}
%\end{table}
%
%

