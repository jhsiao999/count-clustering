\clearpage
\begin{table}[htp]
\caption{Cluster Annotations GTEx V6 data} \label{tab:tab1}
\begin{center}
\begin{tabular}{|c|p{0.5in}|p{1.4in}|p{3.6in}|} 
\hline
Cluster & Gene names & Proteins  & Summary \\
\hline

 \multirow{3}{4em}{\small{cluster red (Nerve, Adipose)} } &  \small{FABP4} & \footnotesize{ fatty acid binding protein 4, adipocyte} & \scriptsize{FABP4 encodes the fatty acid binding protein found in adipocytes, roles include fatty acid uptake, transport, and metabolism}\\ 
 					      & \small{APOD} & \footnotesize{apolipoprotein D} & \scriptsize{encodes a component of high density lipoprotein that has no marked similarity to other apolipoprotein sequences, closely associated with lipoprotein metabolism.}\\
					      & \small{PLIN1} & \footnotesize{perilipin} 1 & \scriptsize{coats lipid storage droplets in adipocytes, thereby protecting them until they can be broken down by hormone-sensitive lipase.} \\
 \hline
 \multirow{3}{4em}{\small{cluster blue (Arteries, Esophagus)} } & \small{MYH11} & \footnotesize{myosin, heavy chain 11, smooth muscle} & \scriptsize{functions as a major contractile protein, converting chemical energy into mechanical energy through the hydrolysis of ATP.} \\
 					    & \small{ACTA2} & \footnotesize{actin, alpha 2, smooth muscle, aorta} & \scriptsize{protein encoded by this gene belongs to the actin family of proteins, which are highly conserved proteins that play a role in cell motility, structure and integrity, defects in this gene cause aortic aneurysm familial thoracic type 6.}\\
					    &  \small{ACTG2} & \footnotesize{actin, gamma 2, smooth muscle, enteric} & \scriptsize{encodes actin gamma 2; a smooth muscle actin found in enteric tissues, involved in various types of cell motility and in the maintenance of the cytoskeleton. }\\
 \hline
 \multirow{3}{4em}{\small{cluster cornflowerblue (Brain)}} & \small{MBP} & \footnotesize{myelin basic protein} & \scriptsize{major constituent of the myelin sheath of oligodendrocytes and Schwann cells in the nervous system} \\
 					    & \small{GFAP} & \footnotesize{glial fibrillary acidic protein} & \scriptsize{encodes one of the major intermediate filament proteins of mature astrocytes, mutations casuses Alexander disease.} \\
					    & \small{SNAP25}  & \footnotesize{synaptosomal-associated protein, 25kDa} & \scriptsize{this gene product is a presynaptic plasma membrane protein involved in the regulation of neurotransmitter release.}\\
\hline
 \multirow{3}{4em}{\small{cluster black (Testis)}} & \small{PRM2} & \footnotesize{protamine 2} & \scriptsize{Protamines are the major DNA-binding proteins in the nucleus of sperm} \\
 					      & \small{PRM1} & \footnotesize{protamine 1} & \scriptsize{Protamines are the major DNA-binding proteins in the nucleus of sperm}  \\
					      & \small{PHF7} & \footnotesize{PHD finger protein 7} & \scriptsize{This gene is expressed in the testis in Sertoli cells but not germ cells, regulates spermatogenesis.} \\
\hline					      					      
 \multirow{3}{4em}{\small{cluster cyan (Thyroid, Stomach}} & \small{TG} & \footnotesize{thyroglobulin} & \scriptsize{thyroglobulin produced predominantly in thyroid gland, synthesizes thyroxine and triiodothyronine, associated with Graves disease and Hashimotot thyroiditis.} \\
 					     & \small{LIPF} & \footnotesize{lipase, gastric} & \scriptsize{encodes gastric lipase, an enzyme involved in the digestion of dietary triglycerides in the gastrointestinal tract, and responsible for 30 $\%$ of fat digestion processes occurring in human.} \\
					     & \small{PGC} & \footnotesize{progastricsin (pepsinogen C)} & \scriptsize{encodes an aspartic proteinase that belongs to the peptidase family A1. The encoded protein is a digestive enzyme that is produced in the stomach and constitutes a major component of the gastric mucosa,  associated with susceptibility to gastric cancers.}\\					     
 \hline
 \multirow{3}{4em}{\small{cluster dark blue (Skin)}} & \small{KRT10} & \footnotesize{keratin 10, type I} & \scriptsize{encodes a member of the type I (acidic) cytokeratin family, mutations associated with epidermolytic hyperkeratosis.}\\
 					    &  \small{KRT1} & \footnotesize{keratin 1, type II} & \scriptsize{specifically expressed in the spinous and granular layers of the epidermis with family member KRT10 and mutations in these genes have been associated with bullous congenital ichthyosiform erythroderma.} \\
					    & \small{KRT2} & \footnotesize{keratin 2, type II} & \scriptsize{expressed largely in the upper spinous layer of epidermal keratinocytes and mutations in this gene have been associated with bullous congenital ichthyosiform erythroderma.}\\
 \hline
 \multirow{3}{4em}{\small{cluster brown4 (Cells fibroblasts)}} & \small{FN1}  & \footnotesize{fibronectin 1} & \scriptsize{Fibronectin is involved in cell adhesion, embryogenesis, blood coagulation, host defense, and metastasis.} \\
 					      & \small{COL1A1} & \footnotesize{collagen, type I, alpha 1} & \scriptsize{Mutations in this gene associated with osteogenesis imperfecta types I-IV, Ehlers-Danlos syndrome type and Classical type, Caffey Disease}. \\
					      & \small{COL1A2} & \footnotesize{collagen, type I, alpha 2} & \scriptsize{Mutations in this gene associated with osteogenesis imperfecta types I-IV, Ehlers-Danlos syndrome type and Classical type, Caffey Disease}. \\
\hline
\end{tabular}
\end{center}
\end{table}


%
%
%\begin{table}
%\newpage
%\begin{center}
%\begin{tabular}{|c|p{0.5in}|p{1.9in}|p{2.5in}|} 
%\hline
%Cluster & Gene names & Proteins  & Summary \\
%\hline
%
% \multirow{3}{4em}{cluster black (Testis)} & PRM2 & protamine 2 & \small{Protamines are the major DNA-binding proteins in the nucleus of sperm} \\
% 					      & PRM1 & protamine 1 & \small{Protamines are the major DNA-binding proteins in the nucleus of sperm}  \\
%					      & PHF7 & PHD finger protein 7 & \small{This gene is expressed in the testis in Sertoli cells but not germ cells, regulates spermatogenesis.} \\
%\hline					      					      
% \multirow{3}{4em}{cluster light blue (Thyroid, Stomach} & TG & thyroglobulin & \small{thyroglobulin produced predominantly in thyroid gland, synthesizes thyroxine and triiodothyronine, associated with Graves disease and Hashimotot thyroiditis.} \\
% 					     & LIPF & lipase, gastric & \small{encodes gastric lipase, an enzyme involved in the digestion of dietary triglycerides in the gastrointestinal tract, and responsible for 30 $\%$ of fat digestion processes occurring in human.} \\
%					     & PGC & progastricsin (pepsinogen C) & \small{encodes an aspartic proteinase that belongs to the peptidase family A1. The encoded protein is a digestive enzyme that is produced in the stomach and constitutes a major component of the gastric mucosa,  associated with susceptibility to gastric cancers.}\\					     
% \hline
% \multirow{3}{4em}{cluster deep blue (Skin)} & KRT10 & keratin 10, type I & \small{encodes a member of the type I (acidic) cytokeratin family, mutations associated with epidermolytic hyperkeratosis.}\\
% 					    &  KRT1 & keratin 1, type II & \small{specifically expressed in the spinous and granular layers of the epidermis with family member KRT10 and mutations in these genes have been associated with bullous congenital ichthyosiform erythroderma.} \\
%					    & KRT2 & keratin 2, type II & \small{expressed largely in the upper spinous layer of epidermal keratinocytes and mutations in this gene have been associated with bullous congenital ichthyosiform erythroderma.}\\
% \hline
% \multirow{3}{4em}{cluster dark brown (Cells fibroblasts)} & FN1  & fibronectin 1 & \small{Fibronectin is involved in cell adhesion, embryogenesis, blood coagulation, host defense, and metastasis.} \\
% 					      & COL1A1 & collagen, type I, alpha 1 & \small{Mutations in this gene associated with osteogenesis imperfecta types I-IV, Ehlers-Danlos syndrome type and Classical type, Caffey Disease}. \\
%					      & COL1A2 & collagen, type I, alpha 2 & \small{Mutations in this gene associated with osteogenesis imperfecta types I-IV, Ehlers-Danlos syndrome type and Classical type, Caffey Disease}. \\
%\hline	
% \end{tabular}
% \end{center}
%\end{table}
%




\newpage
\begin{table}
\begin{center}
\begin{tabular}{|c|p{0.5in}|p{1.4in}|p{3.6in}|}
\hline
Cluster & Gene names & Proteins  & Summary \\
\hline
\multirow{3}{4em}{\small{cluster burlywood (Lung)}} & \small{SFTPB} & \footnotesize{surfactant protein B} & \scriptsize {an amphipathic surfactant protein essential for lung function and homeostasis after birth, muttaions cause pulmonary alveolar proteinosis, fatal respiratory distress in the neonatal period.} \\
 					    & \small{SFTPA2} & \footnotesize{surfactant protein A2} & \scriptsize{Mutations in this gene and a highly similar gene located nearby, which affect the highly conserved carbohydrate recognition domain, are associated with idiopathic pulmonary fibrosis.} \\
					    & \small{SFTPA1} & \footnotesize{surfactant protein A1} &  \scriptsize{encodes a lung surfactant protein that is a member of C-type lectins called collectins, associated with idiopathic pulmonary fibrosis.} \\
\hline				   
 \multirow{3}{4em}{\small{cluster darkgoldenrod (Muscle skeletal)}} & \small{MYH1} & \footnotesize{myosin, heavy chain 1, skeletal muscle, adult }& \scriptsize{a major contractile protein which converts chemical energy into mechanical energy through the hydrolysis of ATP.} \\
 					    & \small{NEB} & \footnotesize{nebulin} & \scriptsize{encodes nebulin, a giant protein component of the cytoskeletal matrix that coexists with the thick and thin filaments within the sarcomeres of skeletal muscle, associated with recessive nemaline myopathy.} \\
					    & \small{MYH2} & \footnotesize{myosin, heavy chain 2, skeletal muscle, adult} & \scriptsize{encodes a member of the class II or conventional myosin heavy chains, and functions in skeletal muscle contraction.} \\
\hline					    
 \multirow{3}{4em}{\small{cluster darkgray (Whole Blood)}} & \small{HBB} & \footnotesize{hemoglobin, beta} & \scriptsize{mutant beta globin causes sickle cell anemia, absence of beta chain/ reduction in beta globin leads to thalassemia.}\\
 					      & \small{HBA2} & \footnotesize{hemoglobin, alpha 2} & \scriptsize{deletion of alpha genes may lead to alpha thalassemia.}  \\
					      & \small{HBA1} & \footnotesize{hemoglobin, alpha 1} & \scriptsize{deletion of alpha genes may lead to alpha thalassemia.}  \\
\hline				      

 \multirow{3}{4em}{\small{cluster deepskyblue (Heart)}} &  \small{NPPA} & \footnotesize{natriuretic peptide A} & \scriptsize{protein encoded by this gene belongs to the natriuretic peptide family, associated with atrial fibrillation familial type 6.} \\
 					      &  \small{MYH6} & \footnotesize{myosin, heavy chain 6, cardiac muscle, alpha} & \scriptsize{encodes the alpha heavy chain subunit of cardiac myosin, mutations in this gene cause familial hypertrophic cardiomyopathy and atrial septal defect 3.} \\
					      &  \small{ACTC1} & \footnotesize{actin, alpha, cardiac muscle 1} & \scriptsize{protein encoded by this gene belongs to the actin family, associated with idiopathic dilated cardiomyopathy (IDC) and familial hypertrophic cardiomyopathy (FHC).} \\
\hline			      
 \multirow{3}{4em}{\small{cluster dark khaki (Esophagus mucosa)}} &  \small{KRT13} & \footnotesize{keratin 13, type I} & \scriptsize{protein encoded by this gene is a member of the keratin gene family, associated with the autosomal dominant disorder White Sponge Nevus.}\\
 					      &  \small{KRT4} & \footnotesize{keratin 4, type II} & \scriptsize{protein encoded by this gene is a member of the keratin gene family, associated with White Sponge Nevus, characterized by oral, esophageal, and anal leukoplakia.} \\
					      &  \small{CRNN} & \footnotesize{cornulin} & \scriptsize{may play a role in the mucosal/epithelial immune response and epidermal differentiation. } \\
\hline
 \multirow{3}{4em}{\small{cluster firebrick (Pancreas)}} &  \small{PRSS1} & \footnotesize{protease, serine 1} & \scriptsize{secreted by pancreas, associated with pancreatitis}\\
 					      &  \small{CPA1} & \footnotesize{carboxypeptidase A1} & \scriptsize{secreted by pancreas, linked to pancreatitis and pancreatic cancer} \\
					      &  \small{PNLIP} & \footnotesize{pancreatic lipase} & \scriptsize{encodes a carboxyl esterase that hydrolyzes insoluble, emulsified triglycerides, and is essential for the efficient digestion of dietary fats. This gene is expressed specifically in the pancreas.}\\
\hline				      
 \multirow{3}{4em}{\small{cluster dark orchid (Liver)}} &  \small{MUC7} & \footnotesize{mucin 7, secreted} & \scriptsize{encodes a small salivary mucin, thought to play a role in facilitating the clearance of bacteria in the oral cavity and to aid in mastication, speech, and swallowing, associated with susceptibility to asthma.} \\
 					      &  \small{ALB} & \footnotesize{albumin} & \scriptsize{functions primarily as a carrier protein for steroids, fatty acids, and thyroid hormones and plays a role in stabilizing extracellular fluid volume.} \\
					      &  \small{HP} & \footnotesize{haptoglobin} & \scriptsize{encodes a preproprotein, which subsequently  produces haptoglobin, linked to diabetic nephropathy, Crohn's disease, inflammatory disease behavior and reduced incidence of Plasmodium falciparum malaria.}\\
\hline					      
 \multirow{3}{4em}{\small{cluster hotpink (Pituitary)}}&  \small{PRL} & \footnotesize{prolactin 2} & \scriptsize{encodes the anterior pituitary hormone prolactin. This secreted hormone is a growth regulator for many tissues, including cells of the immune system.} \\
 					      &  \small{GH1} & \footnotesize{growth hormone 1} & \scriptsize{expressed in the pituitary, play an important role in growth control, mutations in or deletions of the gene lead to growth hormone deficiency and short stature.}\\
					      &  \small{POMC} & \footnotesize{proopiomelanocortin} & \scriptsize{synthesized mainly in corticotroph cells of the anterior pituitary, mutations in this gene have been associated with early onset obesity, adrenal insufficiency, and red hair pigmentation.} \\
\hline
 \end{tabular}
 \end{center}
\end{table}

\clearpage

\subsection{Supplementary Table 1}

\begin{table}[htp]
\begin{center}
\begin{tabular}{|c|p{0.5in}|p{1.4in}|p{3.6in}|} 
\hline
Cluster & Gene names & Proteins  & Summary \\
\hline
 \multirow{3}{4em}{\small{cluster 1,red}}  &  \small{ATP1A2} & \footnotesize{ ATPase, Na+/K+ transporting, alpha 2 polypeptide} & \scriptsize{responsible for establishing and maintaining the electrochemical gradients of Na and K ions across the plasma membrane, mutations in this gene result in familial basilar or hemiplegic migraines, and in a rare syndrome known as alternating hemiplegia of childhood.}   \\ 
 					      & \small{CLU} &  \footnotesize{clusterin} & \scriptsize{protein encoded by this gene is a secreted chaperone that can under some stress conditions also be found in the cell cytosol, also involved in cell death, tumor progression, and neurodegenerative disorders.} \\
					      & \small{DNAJB1} & \footnotesize{DnaJ (Hsp40) homolog, subfamily B, member 1} & \scriptsize{encodes a member of the DnaJ or Hsp40 (heat shock protein 40 kD) family of proteins, that stimulates the ATPase activity of Hsp70 heat-shock proteins  to promote protein folding and prevent misfolded protein aggregation.} \\
\hline
 \multirow{3}{4em}{\small{cluster 2, green}} & \small{SNAP25} & \footnotesize{synaptosomal-associated protein, 25kDa} & \scriptsize{Synaptic vesicle membrane docking and fusion is mediated by SNAREs located on the vesicle membrane (v-SNAREs) and the target membrane (t-SNAREs), involved in the regulation of neurotransmitter release.}\\
 					    & \small{ENO2} & \footnotesize{enolase 2 (gamma, neuronal)} & \scriptsize{ encodes one of the three enolase isoenzymes found in mammals, is found in mature neurons and cells of neuronal origin.} \\
					    &  \small{CHGB} &  \footnotesize{chromogranin B} & \scriptsize{ encodes a tyrosine-sulfated secretory protein abundant in peptidergic endocrine cells and neurons. This protein may serve as a precursor for regulatory peptides.} \\
 \hline
 
 \multirow{3}{4em}{\small{cluster 3,blue}}  &  \small{CALM3} & \footnotesize{calmodulin 3 (phosphorylase kinase, delta)} & \scriptsize{ is a calcium binding protein that plays a role in signaling pathways, cell cycle progression and proliferation.}   \\ 
 					      & \small{FBXL16} &  \footnotesize{F-box and leucine-rich repeat protein 16} & \scriptsize{ Members of the F-box protein family, such as FBXL16, are characterized by an approximately 40-amino acid F-box motif.} \\
					      & \small{UCHL1} & \footnotesize{ubiquitin carboxyl-terminal esterase L1} & \scriptsize{specifically expressed in the neurons and in cells of the diffuse neuroendocrine system. Mutations in this gene may be associated with Parkinson disease.} \\
 \hline
 \multirow{3}{4em}{\small{cluster 4, cyan}} & \small{MBP} & \footnotesize{myelin basic protein} & \scriptsize{protein encoded is a major constituent of the myelin sheath of oligodendrocytes and Schwann cells in the nervous system.} \\
 					    & \small{MYH11} & \footnotesize{glial fibrillary acidic protein} & \scriptsize{ encodes major intermediate filament proteins of mature astrocytes, a marker to distinguish astrocytes during development, mutations in this gene cause Alexander disease, a rare disorder of astrocytes in central nervous system.} \\
					    & \small{ACTA2}  & \footnotesize{secreted protein, acidic, cysteine-rich (osteonectin)}  & \scriptsize{encodes a cysteine-rich acidic matrix-associated protein, required for the collagen in bone to become calcified, in extracellular matrix synthesis and cell shape promotion, associated with tumor suppression.}\\
\hline
\end{tabular}
 \end{center} \label{tab:tab2}
\end{table}



%For the reads expression data, we got the following table
%
%\begin{table}
%\newpage
%\begin{center}
%\begin{tabular}{|c|c|p{1.9in}|p{2.5in}|} 
%\hline
%Cluster & Gene names & Proteins  & Summary \\
%\hline
%
% \multirow{3}{4em}{cluster 1,red (adipose, breast)}  &  ENSG00000170323 & fatty acid binding protein 4, adipocyte & encodes the fatty acid binding protein found in adipocytes, involved in fatty acid uptake, transport, and metabolism.  \\ 
% 					      & ENSG00000181092 & adiponectin, C1Q and collagen domain containing & expressed in adipose tissue exclusively, mutations in this gene are associated with adiponectin deficiency\\
%					      & ENSG00000189058 & apolipoprotein D & encodes a component of high density lipoprotein that has no marked similarity to other apolipoprotein sequences, associated with lipoprotein metabolism. \\
% \hline
% \multirow{3}{4em}{cluster 2, blue (testis, ovary, prostrate)} & ENSG00000122304 & protamine 2 & substitute for histones in the chromatin of sperm during the haploid phase of spermatogenesis, and are the major DNA-binding proteins in the nucleus of sperm in many vertebrates. \\
% 					    & ENSG00000167751 & kallikrein-related peptidase 2 & primarily expressed in prostatic tissue and is responsible for cleaving pro-prostate-specific antigen into its enzymatically active form; may be a prognostic maker for prostate cancer risk \\
%					    &  ENSG00000010318 & PHD finger protein 7 & This gene is expressed in the testis in Sertoli cells but not germ cells, has been implicated in the transcriptional regulation of spermatogenesis\\
% \hline
% \multirow{3}{4em}{cluster 3, shallow blue (colon and esophagus)} & ENSG00000198804 & cytochrome c oxidase subunit I & NA \\
% 					    & ENSG00000198888 & NADH dehydrogenase, subunit 1 (complex I) & NA.\\
%					    & ENSG00000198727  & cytochrome b & NA. \\
%\hline
%\end{tabular}
% \end{center}
% \end{table}
% 
% 
%\begin{table}
%\newpage
%\begin{center}
%\begin{tabular}{|c|c|p{1.9in}|p{2.5in}|} 
%\hline
%Cluster & Gene names & Proteins  & Summary \\
%\hline
%
% \multirow{3}{4em}{cluster 4, black (brain)} & ENSG00000163017  & actin, gamma 2, smooth muscle, enteric & actins are highly conserved proteins that are involved in various types of cell motility and in the maintenance of the cytoskeleton: beta and gamma actins co-exist in most cell types as components of the cytoskeleton and as mediators of internal cell motility. \\
% 					     & ENSG00000133392 & myosin, heavy chain 11, smooth muscle & functions as a major contractile protein, converting chemical energy into mechanical energy through the hydrolysis of ATP\\
%					     & ENSG00000065534 & myosin light chain kinase & encodes myosin light chain kinase which is a calcium/calmodulin dependent enzyme, functions to stabilize unphosphorylated myosin filaments \\	
%\hline				     
% \multirow{3}{4em}{cluster 5,light blue (artery)} & ENSG00000197971 & myelin basic protein & major constituent of the myelin sheath of oligodendrocytes and Schwann cells in the nervous system \\.
% 					     & ENSG00000131095 & glial fibrillary acidic protein & encodes one of the major intermediate filament proteins of mature astrocytes, mutations casuses Alexander disease. \\
%					     & ENSG00000172179 & prolactin & encodes the anterior pituitary hormone prolactin, secreted hormone is a growth regulator for many tissues, including cells of the immune system, may also play a role in cell survival by suppressing apoptosis, and essential for lactation\\					     
% \hline
% \multirow{3}{4em}{cluster 6,deep blue (muscle heart)} & ENSG00000143632 & actin, alpha 1, skeletal muscle & produces highly conserved proteins that play a role in cell motility, structure and integrity, mutations cause nemaline myopathy type 3, congenital myopathy, diseases  leading to muscle fibre defects \\
% 					    &  ENSG00000104879 & creatine kinase, muscle & protein encoded is cytoplasmic enzyme involved in energy homeostasis and serum marker for myocardial infarction. \\
%					    & ENSG00000092054 & myosin, heavy chain 7, cardiac muscle, beta & This gene encodes the beta (or slow) heavy chain subunit of cardiac myosin. It is expressed predominantly in normal human ventricle, also expressed in skeletal muscle tissues rich in slow-twitch type I muscle fibers. Mutations in this gene are associated with familial hypertrophic cardiomyopathy, myosin storage myopathy, dilated cardiomyopathy, and Laing early-onset distal myopathy. \\
% \hline
% \multirow{3}{4em}{cluster 7,dark brown (brain)} & ENSG00000168878 & surfactant protein B & encodes the pulmonary-associated surfactant protein B (SPB), an amphipathic surfactant protein essential for lung function and homeostasis after birth, creted by the alveolar cells of the lung and maintains the stability of pulmonary tissue by reducing the surface tension of fluids that coat the lung, mutations associated with fatal respiratory distress in the neonatal period.\\
% 					    & ENSG00000185303 & surfactant protein A2 & encoding pulmonary-surfactant associated proteins (SFTPA) located on chromosome 10. Mutations in this gene and a highly similar gene located nearby, which affect the highly conserved carbohydrate recognition domain, are associated with idiopathic pulmonary fibrosis.\\
%					    & ENSG00000168484 & surfactant protein C & encodes the pulmonary-associated surfactant protein C (SPC), an extremely hydrophobic surfactant protein essential for lung function and homeostasis after birth, associated with interstitial lung disease in older infants, children, and adults\\
%\hline	
% \end{tabular}
% \end{center}
%\end{table}
%
%
%\begin{table}
%\newpage
%\begin{center}
%\begin{tabular}{|c|c|p{1.9in}|p{2.5in}|} 
%\hline
%Cluster & Gene names & Proteins  & Summary \\
%\hline
%
%		    
% \multirow{3}{4em}{cluster 8, shallow yellow (skin stomach)} & ENSG00000186395 & keratin 10, type I & encodes a member of the type I (acidic) cytokeratin family, mutations associated with epidermolytic hyperkeratosis. \\
% 					    & ENSG00000167768 & keratin 1, type II  & specifically expressed in the spinous and granular layers of the epidermis with family member KRT10 and mutations in these genes have been associated with bullous congenital ichthyosiform erythroderma. \\
%					    & ENSG00000171195 & mucin 7, secreted & encodes a small salivary mucin, which is thought to play a role in facilitating the clearance of bacteria in the oral cavity and to aid in mastication, speech, and swallowing, associated with susceptibility to asthma.\\
%\hline				   
% \multirow{3}{4em}{cluster 9, yellow (cell EBV)} & ENSG00000096088 & progastricsin (pepsinogen C)& encodes an aspartic proteinase that belongs to the peptidase family A1. The encoded protein is a digestive enzyme that is produced in the stomach and constitutes a major component of the gastric mucosa  \\
% 					    & ENSG00000182333 & lipase, gastric & encodes gastric lipase, an enzyme involved in the digestion of dietary triglycerides in the gastrointestinal tract, and responsible for 30 $\%$ of fat digestion processes occurring in human\\
%					    & ENSG00000229859 & pepsinogen 3, group I (pepsinogen A) & encodes a protein precursor of the digestive enzyme pepsin, a member of the peptidase A1 family of endopeptidases, biomarker for atrophic gastritis and gastric cancer \\
%\hline					    
%\multirow{3}{4em}{cluster 11,cyan cluster (cells fibroblasts)} & ENSG00000115414 & fibronectin 1 & Fibronectin is involved in cell adhesion, embryogenesis, blood coagulation, host defense, and metastasis \\
% 					      & ENSG00000108821 & collagen, type I, alpha 1 & Mutations in this gene associated with osteogenesis imperfecta types I-IV, Ehlers-Danlos syndrome type and Classical type, Caffey Disease \\
%					      & ENSG00000164692 & collagen, type I, alpha 2 & Same as above \\
% \hline
% \multirow{3}{4em}{cluster 11,cyan cluster (cells fibroblasts)} & ENSG00000115414 & fibronectin 1 & Fibronectin is involved in cell adhesion, embryogenesis, blood coagulation, host defense, and metastasis \\
% 					      & ENSG00000108821 & collagen, type I, alpha 1 & Mutations in this gene associated with osteogenesis imperfecta types I-IV, Ehlers-Danlos syndrome type and Classical type, Caffey Disease \\
%					      & ENSG00000164692 & collagen, type I, alpha 2 & Same as above \\
%\hline		
% \end{tabular}
% \end{center}
%\end{table}
%
%
