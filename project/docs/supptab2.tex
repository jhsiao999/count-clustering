\section{Supplementary Table 2}
\begin{table}[htp]
\begin{center}
\caption{Cluster Annotations GTEx V6 Brain data (with top gene summaries).\label{tab:supptab2}}
\begin{tabular}{|p{0.7in}|p{0.7in}|p{1.4in}|p{3.6in}|} 
\hline
Cluster & Top Driving \qquad Genes & Gene names & Gene Summary \\
\hline
 \multirow{3}{4em}{\small{cluster 1, royal blue}}  &  \small{\textit{ATP1A2}} & \footnotesize{ ATPase, Na+/K+ transporting, alpha 2 polypeptide} & \scriptsize{responsible for establishing and maintaining the electrochemical gradients of Na and K ions across the plasma membrane, mutations in this gene result in familial basilar or hemiplegic migraines, and in a rare syndrome known as alternating hemiplegia of childhood.}   \\ 
 					      & \small{\textit{CLU}} &  \footnotesize{clusterin} & \scriptsize{protein encoded by this gene is a secreted chaperone that can under some stress conditions also be found in the cell cytosol, also involved in cell death, tumor progression, and neurodegenerative disorders.} \\
					      & \small{\textit{PPP1R1B}} & \footnotesize{protein phosphatase 1 regulatory inhibitor subunit 1B} & \scriptsize{encodes a bifunctional signal transduction molecule, may serve as a therapeutic target for neurologic and psychiatric disorders.} \\
\hline
 \multirow{3}{4em}{\small{cluster 2, yellow orange}} & \small{\textit{PKD1}} & \footnotesize{polycystin 1, transient receptor potential channel interacting} & \scriptsize{functions as a regulator of calcium permeable cation channels and intracellular calcium homoeostasis. It is also involved in cell-cell/matrix interactions and may modulate G-protein-coupled signal-transduction pathways.}\\
 					    & \small{\textit{CBLN3}} & \footnotesize{cerebellin 3 precursor} & \scriptsize{ contain a cerebellin motif and C-terminal C1q signature domain that mediates trimeric assembly of atypical collagen complexes} \\
					    &  \small{\textit{CHGB}} &  \footnotesize{chromogranin B} & \scriptsize{ encodes a tyrosine-sulfated secretory protein abundant in peptidergic endocrine cells and neurons. This protein may serve as a precursor for regulatory peptides.} \\
 \hline
  \multirow{3}{4em}{\small{cluster 3, turquoise}}  & \small{\textit{ENC1}} & \footnotesize{ectodermal-neural cortex 1} & \scriptsize{plays a role in the oxidative stress response as a regulator of the transcription factor Nrf2, may play role in malignant transformation.} \\
 							&  \small{\textit{CALM2}} & \footnotesize{calmodulin 2 (phosphorylase kinase, delta)} & \scriptsize{ is a calcium binding protein that plays a role in signaling pathways, cell cycle progression and proliferation.}   \\ 
 					      & \small{\textit{MAP1A}} &  \footnotesize{microtubule associated protein 1A} & \scriptsize{ involved in microtubule assembly, which is an essential step in neurogenesis,  almost exclusively expressed in the brain.} \\
 \hline
 \multirow{3}{4em}{\small{cluster 4,  red}} & \small{\textit{MBP}} & \footnotesize{myelin basic protein} & \scriptsize{protein encoded is a major constituent of the myelin sheath of oligodendrocytes and Schwann cells in the nervous system.} \\
 					    & \small{\textit{GFAP}} & \footnotesize{glial fibrillary acidic protein} & \scriptsize{ encodes major intermediate filament proteins of mature astrocytes, a marker to distinguish astrocytes during development, mutations in this gene cause Alexander disease, a rare disorder of astrocytes in central nervous system.} \\
					    & \small{\textit{TF}}  & \footnotesize{transferrin}  & \scriptsize{transport iron from the intestine, reticuloendothelial system, and liver parenchymal cells to all proliferating cells in the body, involved in the removal of certain organic matter and allergens from serum.}\\
\hline
\end{tabular}
 \end{center} 
\end{table}

